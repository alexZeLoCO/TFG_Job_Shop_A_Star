%%%%%%%%%%%%%%%%%%%%%%%%%%%%%%%%%%%%%%%%%
% Masters/Doctoral Thesis 
% LaTeX Template
% Version 2.5 (27/8/17)
%
% This template was downloaded from:
% http://www.LaTeXTemplates.com
%
% Version 2.x major modifications by:
% Vel (vel@latextemplates.com)
%
% This template is based on a template by:
% Steve Gunn (http://users.ecs.soton.ac.uk/srg/softwaretools/document/templates/)
% Sunil Patel (http://www.sunilpatel.co.uk/thesis-template/)
%
% Template license:
% CC BY-NC-SA 3.0 (http://creativecommons.org/licenses/by-nc-sa/3.0/)
%
% // ========= //
%
% This template has been partially modified by Alejandro Rodríguez López
%
%%%%%%%%%%%%%%%%%%%%%%%%%%%%%%%%%%%%%%%%%

%----------------------------------------------------------------------------------------
%	PACKAGES AND OTHER DOCUMENT CONFIGURATIONS
%----------------------------------------------------------------------------------------

\documentclass[
11pt, % The default document font size, options: 10pt, 11pt, 12pt
oneside, % Two side (alternating margins) for binding by default, uncomment to switch to one side
spanish, % TODO: Set language [english, spanish, ngerman]
singlespacing, % Single line spacing, alternatives: onehalfspacing or doublespacing
%draft, % Uncomment to enable draft mode (no pictures, no links, overfull hboxes indicated)
%nolistspacing, % If the document is onehalfspacing or doublespacing, uncomment this to set spacing in lists to single
%liststotoc, % Uncomment to add the list of figures/tables/etc to the table of contents
%toctotoc, % Uncomment to add the main table of contents to the table of contents
%parskip, % Uncomment to add space between paragraphs
%nohyperref, % Uncomment to not load the hyperref package
headsepline, % Uncomment to get a line under the header
%chapterinoneline, % Uncomment to place the chapter title next to the number on one line
%consistentlayout, % Uncomment to change the layout of the declaration, abstract and acknowledgements pages to match the default layout
]{MastersDoctoralThesis} % The class file specifying the document structure

\usepackage[utf8]{inputenc} % Required for inputting international characters
\usepackage[T1]{fontenc} % Output font encoding for international characters

\usepackage{myconfig} % Local file, contains command configurations

\setcounter{secnumdepth}{7}

\usepackage{mathpazo} % Use the Palatino font by default
\usepackage[
  backend=bibtex,
  style=alphabetic,
  sorting=ynt,
  natbib=true,
  language=spanish
]{biblatex} % Use the bibtex backend with the authoryear citation style (which resembles APA)
\addbibresource{bibliography.bib} % The filename of the bibliography

\usepackage[autostyle=true]{csquotes} % Required to generate language-dependent quotes in the bibliography

%----------------------------------------------------------------------------------------
%	MARGIN SETTINGS
%----------------------------------------------------------------------------------------

\geometry{
	paper=a4paper, % Change to letterpaper for US letter
	inner=2.5cm, % Inner margin
	outer=3.8cm, % Outer margin
	bindingoffset=.5cm, % Binding offset
	top=1.5cm, % Top margin
	bottom=1.5cm, % Bottom margin
	%showframe, % Uncomment to show how the type block is set on the page
}

%----------------------------------------------------------------------------------------
%	THESIS INFORMATION
%----------------------------------------------------------------------------------------

% Should you want to have an href alongside the name, state it inside the variable using:
% \href{url}{name}

% TODO: Your thesis title, print it with \ttitle
\thesistitle{Optimización de Algoritmos de Búsqueda en Grafos:\\Implementación y Comparación de Rendimiento en FPGA}
% TODO: Your tutor name, print it with \tutor
\tutor{Juan José Palacios Alonso} 
% TODO: Your other tutor name, print it with \tutorBis
\tutorBis{} 
% TODO: Your degree name, print it with \degreename
\degree{Grado de Ingeniería Informática en Tecnologías de la Información} 
% TODO: Your name, print it with \authorname
\author{\href{https://www.linkedin.com/in/alex02}{Alejandro Rodríguez López}} 
% TODO: Your peer name, print it with \authornameBis
\authorBis{} 
% TODO: Your address, print it with \addressname
\addresses{33204, Gijón, Asturias, España} 

% TODO: Your subject area, print it with \subjectname
\subject{Trabajo de Fin de Grado} 
% TODO: Keywords for your thesis, print it with \keywordnames
\keywords{Graph, Graph search, algorithm, parallel, FPGA, Python, C, low level, A star} 
% TODO: Your university's name and URL, print it with \univname
\university{Universidad de Oviedo} 
% TODO: Your department's name and URL, print it with \deptname
\department{Informática} 
% TODO: Your research group's name and URL, print it with \groupname
\group{Ciencia de la Computación e Inteligencia Artificial} 
% TODO: Your faculty's name and URL, print it with \facname
\faculty{Escuela Politécnica de Ingeniería de Gijón} 

\AtBeginDocument{
	\hypersetup{pdftitle=\ttitle} % Set the PDF's title to your title
	\hypersetup{pdfauthor=\authorname} % Set the PDF's author to your name
	\hypersetup{pdfkeywords=\keywordnames} % Set the PDF's keywords to your keywords
}

\makeindex[columns=1, options= -s index_style.ist]

\begin{document}

\frontmatter % Use roman page numbering style (i, ii, iii, iv...) for the pre-content pages

\pagestyle{plain} % Default to the plain heading style until the thesis style is called for the body content

%----------------------------------------------------------------------------------------
%	TITLE PAGE
%----------------------------------------------------------------------------------------

\begin{titlepage}
\begin{center}

\vspace*{.06\textheight}
{\scshape\LARGE \univname\\\facname\par}\vspace{1.5cm} % University name
\textsc{\Large \subjectname}\\[0.5cm] % Thesis type

\HRule \\[0.4cm] % Horizontal line
{\huge \bfseries \ttitle\par}\vspace{0.4cm} % Thesis title
\HRule \\[1.5cm] % Horizontal line
 
\begin{minipage}[t]{\textwidth}
	\begin{flushleft} \large
		\emph{Autor:}\\
		\authorname\\
    \vspace{2em}
	\end{flushleft}
\end{minipage}
\begin{minipage}[t]{0.8\textwidth}
	\begin{flushright} \large
		\emph{Tutor:} \\
		\tutorName
	\end{flushright}
\end{minipage}\\
 
\vfill

% University requirement text
\large\textit{\requirements}\\[0.3cm] 
% \textit{en}\\[0.4cm]
% \groupname\\[2cm] % Research group name and department name
 
\vfill

{\large \today}\\[4cm] % Date
%\includegraphics{Logo} % University/department logo - uncomment to place it
 
\vfill
\end{center}
\end{titlepage}

%----------------------------------------------------------------------------------------
%	DECLARATION PAGE
%----------------------------------------------------------------------------------------

%\begin{declaration}
%\addchaptertocentry{\authorshipname} % Add the declaration to the table of contents
%\noindent Yo, \authorname, declaro que este documento titulado, \enquote{\ttitle} y el trabajo presentado en él son de mi propiedad.
%Afirmo que:

%\begin{itemize} 
	%\item Este trabajo fue realizado completa o parcialmente durante mi estancia como estudiante en el grado de \degreename.
	%\item Aquellas partes de este documento que hayan sido previamente publicadas se encuentran debidamente indicadas.
	%\item Aquellas partes de este documento que se apoyen en trabajos previamente publicados se encuentran debidamente indicadas.
	%\item Todas las fuentes utilizadas para la realizacion de este trabajo se encuentran listadas.
%\end{itemize}

%\vfill

%\noindent Signed: \authorname\\
%\rule[0.5em]{25em}{0.5pt} % This prints a line for the signature
 
%\noindent Date: Julio 13, 2023\\
%\rule[0.5em]{25em}{0.5pt} % This prints a line to write the date
%\end{declaration}

%\cleardoublepage

%----------------------------------------------------------------------------------------
%	QUOTATION PAGE
%----------------------------------------------------------------------------------------

%\vspace*{0.2\textheight}

%\noindent\enquote{
	%\itshape Thanks to my solid academic training, today I can write hundreds of words on virtually any topic
	%without possessing a shred of information, which is how I got a good job in journalism.
%}\bigbreak

%\hfill Dave Barry

%----------------------------------------------------------------------------------------
%	ABSTRACT PAGE
%----------------------------------------------------------------------------------------


\begin{abstract}
% TODO: Add the abstract
\addchaptertocentry{\abstractname}
La gran mayoría de ordenadores hoy en día poseen varios núcleos en sus procesadores,
gracias a ellos se nos permite realizar diversas tareas de forma concurrente
sin percatarnos de que un único procesador sólo puede hacer una tarea a la vez.
La mayoría de programas deberían aprovechar este principio para acelerar y
presentar resultados a sus usuarios más rápido.\\

En algunos casos esto no es así, tal vez porque sus desarrolladores no se han dignado
a explorar los beneficios del paralelismo.
Otra posibilidad es que el simple uso de varios hilos no implica una mejora en rendimiento,
el diseño del algoritmo y sus secciones paralelas son de crucial importancia
para obtener beneficios tangibles.\\

Algunos algoritmos son más propensos al paralelismo,
en ellos resulta más fácil hallar zonas en las que el uso de múltiples hilos
es sencillo de implementar y no existe mucho sobrecoste.
Desafortunadamente esto no sucede en todos los algoritmos.
En esta investigación se analiza el algoritmo A*,
un claro ejemplo de cómo el paralelismo requiere
una muy detenida atención a la estrategia para que sea rentable.
\end{abstract}

%----------------------------------------------------------------------------------------
%	ACKNOWLEDGEMENTS
%----------------------------------------------------------------------------------------

%\begin{acknowledgements}
%\addchaptertocentry{\acknowledgementname} % Add the acknowledgements to the table of contents
%The acknowledgments and the people to thank go here, don't forget to include your project advisor\ldots
%\begin{itemize}
	%\item \groupname
		%\begin{itemize}
			%\item \tutorEmpName
			%\item Antiñolo, David Espejo
			%\item Omen, Juan David
			%\item Rico, Luis Rodríguez
		%\end{itemize}
	%\item \facname
		%\begin{itemize}
			%\item \tutorEpiName
		%\end{itemize}
%\end{itemize}
%\end{acknowledgements}

%----------------------------------------------------------------------------------------
%	LIST OF CONTENTS/FIGURES/TABLES PAGES
%----------------------------------------------------------------------------------------

\tableofcontents % Prints the main table of contents

\listoffigures % Prints the list of figures

\listoftables % Prints the list of tables

%----------------------------------------------------------------------------------------
%	ABBREVIATIONS
%----------------------------------------------------------------------------------------

\begin{abbreviations}{ll} % Include a list of abbreviations (a table of two columns)

% TODO: Add Abbreviations

%\textbf{LAH} & \textbf{L}ist \textbf{A}bbreviations \textbf{H}ere\\
%\textbf{WSF} & \textbf{W}hat (it) \textbf{S}tands \textbf{F}or\\
%\textbf{HR} & \textbf{H}uman \textbf{R}esources\\
%\textbf{BI} & \textbf{B}ase \textbf{I}mponible\\
%\textbf{CP} & \textbf{C}aso de \textbf{P}rueba\\
%\textbf{E} & Número de \textbf{E}mpleadores\\
%\textbf{I} & \textbf{I}mpuestos calculados a partir de BI\\
%\textbf{P} & Cantidad pagada en \textbf{P}réstamos por hipotecas de vivienda habitual comprada antes de 2013\\
%\textbf{R} & \textbf{R}etenciones\\
%\textbf{SP} & \textbf{S}ituación de \textbf{P}rueba\\
\textbf{CPU} & \textbf{Central} \textbf{Processing} \textbf{Unit}\\
\textbf{CSV} & \textbf{C}omma \textbf{S}eparated \textbf{V}alue\\
\textbf{FCFS} & \textbf{F}irst \textbf{C}ome \textbf{F}irst \textbf{S}erve\\
\textbf{FPGA} & \textbf{F}ield \textbf{P}rogrammable \textbf{G}ate \textbf{A}rray\\
\textbf{HDA*} & \textbf{H}ash \textbf{D}istributed \textbf{A}*\\
\textbf{HDL} & \textbf{H}ardware \textbf{D}escription \textbf{L}anguage\\
\textbf{HLS} & \textbf{H}igh \textbf{L}evel \textbf{S}ynthesis\\
\textbf{HTTP} & \textbf{H}yper \textbf{T}ext \textbf{T}ransfer \textbf{P}rotocol\\
\textbf{JSP} & \textbf{J}ob \textbf{S}hop \textbf{P}roblem\\
\textbf{JSSP} & \textbf{J}ob \textbf{S}hop \textbf{S}cheduling \textbf{P}roblem\\
\textbf{OS} & \textbf{O}perative \textbf{S}ystem\\
\textbf{SoC} & \textbf{S}ystem on \textbf{C}hip\\

\end{abbreviations}

%----------------------------------------------------------------------------------------
%	PHYSICAL CONSTANTS/OTHER DEFINITIONS
%----------------------------------------------------------------------------------------

%\begin{constants}{lr@{${}={}$}l} % The list of physical constants is a three column table

% The \SI{}{} command is provided by the siunitx package, see its documentation for instructions on how to use it

%Speed of Light & $c_{0}$ & \SI{2.99792458e8}{\meter\per\second} (exact)\\
%Constant Name & $Symbol$ & $Constant Value$ with units\\

%\end{constants}

%----------------------------------------------------------------------------------------
%	SYMBOLS
%----------------------------------------------------------------------------------------

%\begin{symbols}{lll} % Include a list of Symbols (a three column table)

%$a$ & distance & \si{\meter} \\
%$P$ & power & \si{\watt} (\si{\joule\per\second}) \\
%Symbol & Name & Unit \\

%\addlinespace % Gap to separate the Roman symbols from the Greek

%$\omega$ & angular frequency & \si{\radian} \\

%\end{symbols}

%----------------------------------------------------------------------------------------
%	DEDICATION
%----------------------------------------------------------------------------------------

%\dedicatory{For/Dedicated to/To my\ldots} 

%----------------------------------------------------------------------------------------
%	BLANK PAGE
%----------------------------------------------------------------------------------------

%\blankpage

%----------------------------------------------------------------------------------------
%	THESIS CONTENT - CHAPTERS
%----------------------------------------------------------------------------------------

\mainmatter % Begin numeric (1,2,3...) page numbering

\pagestyle{thesis} % Return the page headers back to the "thesis" style
\setlength{\parskip}{\baselineskip}%


% Include the chapters of the thesis as separate files from the Chapters folder
% Uncomment the lines as you write the chapters

\Chapter{Hipótesis de Partida y Alcance}{Estado del Arte}

\section{Objeto de la investigación}
\index{Objeto de la investigación}

El presente proyecto tiene como objetivo principal descubrir los
beneficios del paralelismo aplicado al algoritmo A*.
Para estudiar el rendimiento de las implementaciones desarrolladas
se utilizará el Job Shop Scheduling Problem.
Adicionalmente, se observará el rendimiento de una implementación
de la misma solución en una FPGA.

La motivación principal para la realización de este proyecto
emana en un interés personal por el diseño, implementación y optimización
de algoritmos aplicables a problemas reales.

El Job Shop Scheduling Problem es un problema de optimización
sobre la planificación de horarios.
Este problema en particular es mundialmente conocido,
ha sido resuelto utilizando un gran abanico de algoritmo diferentes
y ha sido profundamente estudiado.

La resolución del Job Shop Scheduling Problem requiere el diseño e implementación de
un algoritmo capaz de recibir como entrada las descripciones de una plantilla
de trabajadores y un listado de trabajos y tareas a realizar puede ser
aplicable en ámbitos industriales donde la automatización de la creación
de planificaciones pueda ser de interés.

\section{Estado del arte}
\index{Estado del arte}

Este proyecto abarca diversos tópicos, cuyas bibliografías
(incluso en individual) tienen una gran extensión.
A pesar de ello, resulta complicado hallar estudios previos sobre implementaciones
paralelas del algoritmo A* enfocadas a la resolución del Job Shop Scheduling Problem
utilizando FPGAs.
Así pues, el punto de partida de este proyecto
se compone principalmente de estudios
sobre los distintos tópicos de forma individual.

El Job Shop Scheduling Problem tiene su origen en la década de 1960,
desde entonces ha sido utilizado frecuentemente (incluso hasta el día de hoy)
como herramienta de medición del rendimiento de algoritmos que sean 
capaces de resolverlo \cite{Man67}.

A lo largo de los años, se han realizado numerosos trabajos con el objetivo
de recoger distintos algoritmos que resuelvan el problema.
Dichos algoritmos provienen de diferentes ámbitos de la computación.
Entre ellos se pueden encontrar búsquedas en grafos,
listas de prioridad, ramificación y poda, algoritmos genéticos,
simulaciones Monte Carlo o
métodos gráficos como diagramas Gantt y Pert.
\cite{Yan77}, \cite{Nil69}, \cite{KTM99}, \cite{BC22}.

El algoritmo A* fue diseñado a finales de la década de 1960
con el objetivo de implementar el enrutamiento de un robot
conocido como \italic{"Shakey the Robot"} \cite{Nil84}.
A* es una evolución del conocido algoritmo de Dijkstra
frecuentemente utilizado también para la búsqueda en grafos.
La principal diferencia entre estos dos algoritmos es el
uso de una función heurística en el A* que `guía' al algoritmo
en la dirección de la solución.
\cite{HNR68}, \cite{MSV13}, \cite{Kon14}.

El algoritmo A* no es propenso a la paralelización,
no existe una implementación simple que aproveche el funcionamiento de
varios procesadores de forma simultanea.
En su lugar existen varias alternativas que paralelizan el algoritmo,
cada una de ellas con sus fortalezas y debilidades.
Estas diferentes versiones serán estudiadas, implementadas y
probadas en esta investigación.
\cite{Zag17}, \cite{WH16}.

El tópico sobre el que menor cantidad de documentación existe
y que supone una mayor curva de aprendizaje es sin lugar a duda
la implementación del algoritmo A* en una FPGA.
A diferencia de este proyecto, la principal fuente de bibliografía
sobre este tópico desarrolla una implementación personalizada del algoritmo
que reporta una aceleración de casi el $400\percentsign$.
\cite{ZJW20}.

\section{Requisitos}
\index{Requisitos}

El algoritmo a desarrollar en esta investigación deberá recibir como entrada
una estructura de datos que contenga los trabajos,
para cada trabajo la lista de tareas que lo compone,
para cada tarea su duración y el listado de trabajadores cualificados para realizarla
(generalmente, la longitud de esta lista será 1).

Como resultado, el algoritmo deberá retornar un listado de trabajos,
para cada trabajo el instante de tiempo en el que se inicia cada una de sus tareas
y un listado con los instantes de tiempo en los que cada trabajador quedará libre.
El máximo elemento del listado de instantes de tiempo en los que cada trabajador queda libre
se define como el \italic{makespan}, el tiempo necesario para finalizar todos los trabajos.

\section{Alcance}
\index{Alcance}

El presente documento describe el problema a resolver en detalle,
los diferentes algoritmos implementados,
observaciones sobre los mismos,
casos de prueba utilizados para obtener métricas de rendimiento
y observaciones sobre las métricas obtenidas.

Entre las observaciones tanto de los algoritmos como de las métricas
obtenidas, se encontrarán razonamientos sobre los resultados así
como explicaciones de las razones por las cuales un algoritmo
presenta un rendimiento distinto a otro.

\Chapter{Problema a investigar}{Análisis, implementación y optimización}

\section{Job Shop Scheduling Problem}
\index{Job Shop Scheduling Problem}

Se estudia la implementación de una solución al problema
\italic{Job Shop Scheduling (JSP)}~\cite{Yan77}
\footnote{El problema también es conocido por otros nombres similares
como \italic{Job Shop Scheduling Problem (JSSP)} o Job Scheduling Problem (JSP).}
resuelto utilizando el algoritmo A*.
Como su nombre indica, se trata de un problema en el que se debe
crear una planificación.
Claramente, el JSP es un problema de optimización.

El JSP busca una planificación para una serie de
máquinas (o trabajadores) que deben realizar un número conocido
de trabajos.
Cada trabajo está formado por una serie de operaciones (o tareas),
con una duración conocida.
Las tareas de un mismo trabajo deben ser ejecutadas en un orden específico.

Existen numerosas variantes de este problema,
algunas permiten la ejecución
en paralelo de algunas tareas o requieren que alguna tarea
en específico sea ejecutada por un trabajador (o tipo de trabajador)
en particular.
Por ello, es clave denotar las normas que se aplicarán a la hora de resolver
el problema JSP\@:

\begin{enumerate}[itemsep=0.25px]
    \item Existen un número natural conocido de trabajos.
    \item Todos los trabajos tienen el mismo número natural conocido de tareas.
    \item A excepción de la primera tarea de cada trabajo,
    todas tienen una única tarea predecesora que debe ser completada
    antes de iniciar su ejecución.
    \item Cada tarea puede tener una duración distinta.
    \item La duración de cada tarea es un número natural conocido.
    \item Existe un número natural conocido de trabajadores.
    \item Cada tarea tiene un trabajador asignado,
    de forma que sólo ese trabajador puede ejecutar la tarea.
    \item Una vez iniciada una tarea, no se puede interrumpir su ejecución.
    \item Un mismo trabajador puede intercalar la ejecución de tareas de diferentes trabajos.
    \item Un trabajador sólo puede realizar una tarea al mismo tiempo.
    \item Los tiempos de preparación de un trabajador antes de realizar una tarea son nulos.
    \item Los tiempos de espera entre la realización de una tarea y otra son nulos.
\end{enumerate}

\section{Método a resolver}

\subsection{Algoritmo}
\index{Algoritmo A*}

Existen numerosos algoritmos capaces de resolver el problema del JSP\@.
Estrategias más simples como listas ordenadas en función de la duración
de los trabajos, algoritmos genéticos, técnicas gráficas,
algoritmos \italic{Branch and Bound} y heurísticos.
En este estudio se utilizará un algoritmo heurístico, A* (\italic{A star})~\cite{HNR68}
para resolver el problema JSP\@.

El A* es una evolución del algoritmo de Dijkstra.
Su principal diferencia es la implementación de una función heurística
que se utiliza para decidir el siguiente nodo a expandir.
De esta forma, se podría decir que el algoritmo A* va `guiado'
hacia la solución, mientras que el algoritmo de Dijkstra
sigue los caminos con menor coste (Véase figura \ref{fig:DijkstraA*Comparison}).

\begin{figure}
\begin{subfigure}{.5\textwidth}
\begin{center}
\begin{tikzpicture}[node distance=1.5cm]
    \node (00) [dot, fill=green!30] {00};
    \node (01) [dot, right of=00, fill=blue!30] {01};
    \node (02) [dot, right of=01] {02};
    \node (03) [dot, right of=02, fill=blue!30] {03};

    \node (10) [dot, below of=00] {10};
    \node (11) [dot, right of=10, fill=blue!30] {11};
    \node (12) [dot, right of=11, fill=blue!30] {12};
    \node (13) [dot, right of=12, fill=blue!30] {13};

    \node (20) [dot, below of=10] {20};
    \node (21) [dot, right of=20] {21};
    \node (22) [dot, right of=21, fill=blue!30] {22};
    \node (23) [dot, right of=22] {23};

    \node (30) [dot, below of=20] {30};
    \node (31) [dot, right of=30, fill=blue!30] {31};
    \node (32) [dot, right of=31, fill=blue!30] {32};
    \node (33) [dot, right of=32, fill=yellow!30] {33};

    \draw [arrow] (00) -- (11);
    \draw [arrow] (00) -- (01);

    \draw [arrow] (11) -- (22);
    \draw [arrow] (11) -- (12);
    \draw [arrow] (12) -- (13);
    \draw [arrow] (13) -- (03);

    \draw [arrow] (22) -- (33);
    \draw [arrow] (22) -- (32);

    \draw [arrow] (32) -- (31);
\end{tikzpicture}
\subcaption{Algoritmo Dijkstra}
\end{center}
\end{subfigure}
\begin{subfigure}{.5\textwidth}
\begin{center}
\begin{tikzpicture}[node distance=1.5cm]
    \node (00) [dot, fill=green!30] {00};
    \node (01) [dot, right of=00] {01};
    \node (02) [dot, right of=01] {02};
    \node (03) [dot, right of=02] {03};

    \node (10) [dot, below of=00] {10};
    \node (11) [dot, right of=10, fill=blue!30] {11};
    \node (12) [dot, right of=11] {12};
    \node (13) [dot, right of=12] {13};

    \node (20) [dot, below of=10] {20};
    \node (21) [dot, right of=20] {21};
    \node (22) [dot, right of=21, fill=blue!30] {22};
    \node (23) [dot, right of=22] {23};

    \node (30) [dot, below of=20] {30};
    \node (31) [dot, right of=30] {31};
    \node (32) [dot, right of=31] {32};
    \node (33) [dot, right of=32, fill=yellow!30] {33};

    \draw [arrow] (00) -- (11);
    \draw [arrow] (11) -- (22);
    \draw [arrow] (22) -- (33);
\end{tikzpicture}
\subcaption{Algoritmo A*}
\end{center}
\end{subfigure}
\caption{Comparativa entre algoritmos Dijkstra y A*}
\label{fig:DijkstraA*Comparison}
\end{figure}

El algoritmo A* utiliza varios componentes para resolver problemas
de optimización. A continuación se describe cada uno de ellos.

\subsubsection{Componentes A*}

\paragraph{Estado}~
\index{A*: Estado}

El estado es una estructura de datos que describe la situación
del problema en un punto determinado.
Estos estados deben ser comparables,
debe ser posible dados dos estados conocer si son iguales o distintos.
En el caso del JSP, el estado podría estar formado por la siguiente información:

\begin{itemize}[itemsep=0.25px]
    \item Instante de tiempo en el que comienza cada tarea planificada.
    \item Instante de tiempo futuro en el que cada trabajador estará libre
    (i.e. finaliza la tarea que estaba realizando).
\end{itemize}

El resultado final del JSP será un estado donde todas las tareas han sido planificadas.

\paragraph{Costes}~

El algoritmo A* utiliza 3 costes distintos para resolver el problema de optimización:

\subparagraph{Coste G}~
\index{A*: Coste G}

El coste G (de ahora en adelante $cost_g$) es el coste desde el estado inicial
hasta el estado actual.
Este coste es calculado buscando el mayor tiempo de fin de
las tareas ya planificadas.

\subparagraph{Coste H}~
\index{A*: Coste H}

El coste H (de ahora en adelante $cost_h$) es el coste estimado desde el estado actual
hasta el estado final.
Este coste es calculado utilizando una función heurística,
que estima el coste.

\subparagraph{Coste F}~
\index{A*: Coste F}

El coste F (de ahora en adelante $cost_f$) es el coste estimado desde el estado inicial
hasta el estado final pasando por el estado actual.

Por lo tanto, \[cost_f = cost_g + cost_h\]

\paragraph{Generación de sucesores}~
\index{A*: Sucesores / Vecinos / Hijos}

El algoritmo A* debe generar un número de estados sucesores dado un estado cualquiera
\footnote{Dependiendo de la implementación, puede existir alguna excepción a esta norma,
como el estado objetivo que puede no tener sucesores.}.
Por lo que será necesario una función que dado un estado retorne un listado de estados.

Dado un estado donde existen $N$ tareas por ejecutar ($T_0 \dots T_N$) y
un trabajador cualquiera sin tarea asignada tendrá $N$ estados sucesores.
En cada uno, el trabajador libre tendrá asignada cada una de las tareas,
desde $T_0$ hasta $T_N$
\footnote{Suponiendo que:
\begin{enumerate}[itemsep=0.25px]
    \item Todas las tareas $T_x \in (T_0 \dots T_N$)
pueden ser ejecutadas por el trabajador libre.
    \item Todas las tareas $T_x \in (T_0 \dots T_N$)
pueden ser ejecutadas en este instante (e.g. cada una es de un trabajo diferente).
\end{enumerate}}.

\paragraph{Listas de prioridad}~
\index{A*: Lista de prioridad}

El algoritmo A* utiliza dos listas de estados: la lista abierta y la lista cerrada.
La lista cerrada contiene los estados que ya han sido estudiados mientras que
la lista abierta contiene los estados que aún están por estudiar.

Cada vez que se estudia un estado de la lista abierta,
se obtienen sus sucesores que son añadidos a la lista abierta
(siempre y cuando no estén en la lista cerrada)
mientras que el estado estudiado pasa a la lista cerrada.

Los estados de la lista abierta están ordenados en función de su $cost_f$,
de menor a mayor.
De esta forma, se tiene acceso inmediato al elemento con menor $cost_f$.

\subsubsection{Pseudocódigo}

\begin{lstlisting}[language=Python]
    lista_abierta = SortedList()
    lista_abierta.append(estado_inicial)

    g_costes = {}
    f_costes = {}

    g_costes[estado_inicial] = 0
    f_costes[estado_inicial] = calcular_h_coste(estado_inicial)

    while (not lista_abierta.empty()):
        estado_actual = lista_abierta.pop()

        if (estado_actual == estado_final):
            return estado_actual

        estados_sucesores = calcular_sucesores(estado_actual)

        for estado_sucesor in estados_sucesores:
            sucesor_g_coste = calcular_g_coste(estado_sucesor)
            if (sucesor_g_coste < g_costes[estado_sucesor]):
                g_costes[estado_sucesor] = sucesor_g_coste
                f_costes[estado_sucesor] = sucesor_g_coste + calcular_h_coste(estado_sucesor)
                if (estado_sucesor not in lista_abierta):
                    lista_abierta.append(estado_sucesor)
            
\end{lstlisting}

\subsection{Equipo de Estudio}

Este algoritmo es implementado y optimizado en diversas arquitecturas.
Posteriormente, se realizan comparaciones entre ellas.

\subsubsection{Arquitectura x86}
\index{Arquitectura x86}

Inicialmente, se realiza una implementación del algoritmo utilizando \Python.
Esta versión permite comprobar rápidamente el correcto funcionamiento del mismo
así como llevar a cabo pruebas rápidas sin necesidad de compilación y
estudiar los posibles cuellos de botella del algoritmo.

Posteriormente, se desarrolla una nueva versión del mismo algoritmo
utilizando C++, un lenguaje compilado, imperativo y orientado a objetos
que facilita la paralelización gracias a librerías como 
\href{https://www.openmp.org/}{OpenMP}\@.

Una vez desarrolladas ambas versiones monohilo,
se comienza la implementación de versiones multihilo
que serán posteriormente comparadas.

\subsubsection{FPGA}
\index{FPGA}

Finalmente, se desarrolla una implementación del algoritmo
diseñado para ser ejecutado en una FPGA\@.
Esta aceleradora, se encuentra embebida en una placa SoC
Zybo Z7 10 acompañada de un procesador ARM\@.

Para realizar esta implementación,
se utiliza el software propio de Xilinx (AMD),
Vitis HDL\@.
Este programa ofrece entre muchas otras herramientas
un sintetizador capaz de transpilar código C++ a Verilog
que puede ser entonces compilado
para ejecutarse en la FPGA\@.

\section{Método de comparativas}

Las comparativas entre las diferentes implementaciones
del algoritmo se realizan en base a varias características.
Principalmente:

\begin{enumerate}[itemsep=0.25px]
    \item Tiempo de ejecución.
    \item Calidad de la solución.
\end{enumerate}

Como es lógico, el algoritmo es ejecutado utilizando
distintos datos de entrada múltiples veces.

\begin{notebox}
    La segunda ejecución de cualquier algoritmo suele tender a
    requerir menos tiempo debido al funcionamiento de la caché.
    
    Para evitar este fenómeno, el algoritmo se ejecuta
    siempre $N$ veces ignorando las métricas de la primera ejecución.
\end{notebox}

\section{Implementación}

Tanto \Python como C++ son lenguajes orientados a objetos.
Esta sección contiene las descripciones de las diferentes
clases diseñadas para dar soporte al algoritmo.

\subsection{Task}
\index{Task}

La clase \lstinline{Task} correspone a una tarea a realizar.
Una instancia de esta clase está definida por los atributos:
\begin{itemize}[itemsep=0.25px]
    \item \lstinline{unsigned int duration}: Duración de la tarea.
    \item \lstinline{std::vector<int> qualified_workers}: Listado de trabajadores que pueden realizar la tarea.
\end{itemize}

\subsection{State}
\index{State}

La clase \lstinline{State} corresponde a un estado (o nodo).
Una instancia de esta clase está definida por los atributos:
\begin{itemize}[itemsep=0.25px]
    \item \lstinline{std::vector<std::vector<Task>> jobs}: Lista de trabajos y tareas a ejecutar.
    \item \lstinline{std::vector<std::vector<int>> schedule}: Planificación actual.
    \item \lstinline{std::vector<int> workers_status}: Instantes en los que cada trabajador queda libre.
\end{itemize}

\begin{notebox}
    Nótese que el atributo \lstinline{std::vector<std::vector<Task>> jobs} será el mismo
    en todos los estados de un mismo problema.
    Por lo que no será necesario revisarlo en
    \lstinline{State::operator==} ni \lstinline{State::operator()}.
    Si el consumo de memoria fuese de importancia,
    sería posible utilizar una referencia para evitar
    almacenar esta estructura múltiples veces.
\end{notebox}

El algoritmo A* requiere que se creen estructuras de datos que contendrán instancias
de la clase \lstinline{State}.
Estas estructuras necesitan que se proporcionen implementaciones para los operadores
\lstinline{State::operator==} y \lstinline{State::operator()} de la clase \lstinline{State}.
Para diseñar las implementaciones de estos operadores
se estudian previamente los atributos que componen la clase \lstinline{State}:

\begin{itemize}[itemsep=0.25px]
    \item \lstinline{std::vector<std::vector<Task>> jobs}: Es igual en todas las
    instancias de \lstinline{State}, por lo que será ignorado.
    \item \lstinline{std::vector<std::vector<int>> schedule}: Proporciona
    información crucial sobre el estado ($cost_g$ y $cost_h$).
    \item \lstinline{std::vector<int> workers_status}: No proporciona
    información alguna sobre los costes,
    pero es necesario para distinguir dos estados diferentes
    ya que es posible que dos estados tengan los mismos costes pero a través de
    planificaciones distintas.
\end{itemize}

Por ello, será necesario definir dos operadores \lstinline{State::operator()}:
uno que sea indiferente al atributo \lstinline{std::vector<int> workers_status}
(\lstinline{StateHash::operator()})
y otro que sí lo tenga en cuenta para distinguir diferentes instancias de \lstinline{State}
(\lstinline{FullHash::operator()}).

\section{Optimización}

\subsection{State}

La operación \lstinline{operator()} es ejecutada varias veces para cada
\lstinline{State}, este método tiene una complejidad de $O(n^2)$,
por lo que su valor se almacena tras calcularlo por primera vez
en un atributo del propio \lstinline{State}.

La operación \lstinline{operator()} contiene 2 bucles \lstinline{for} anidados.
Su principal objetivo es calcular una reducción de los atributos de la instancia
\lstinline{State}.
Se utiliza \lstinline{#pragma omp parallel for collapse(2) reduction(+: seed)}
para paralelizar la reducción.

\begin{lstlisting}[
    language=C++,
    caption=Implementación de \lstinline{FullHash::operator()}
]
std::size_t FullHash::operator()(State key) const
{
    if (key.get_full_hash() != UNINITIALIZED_HASH)
        return key.get_full_hash();
    std::vector<std::vector<int>>
        schedule = key.get_schedule();
    std::vector<int> workers_status = key.get_workers_status();
    std::size_t seed = schedule.size() * schedule[0].size() * workers_status.size();

#pragma omp parallel for reduction(+ : seed)
    for (size_t i = 0; i < workers_status.size(); i++)
        seed += (workers_status[i] * 10 ^ i);

    if (schedule.empty())
        return seed;

    const std::size_t nTasks = schedule[0].size();
#pragma omp parallel for collapse(2) reduction(+ : seed)
    for (size_t i = 0; i < schedule.size(); i++)
    {
        for (size_t j = 0; j < nTasks; j++)
            seed += (schedule[i][j] * 10 ^ ((1 + i + workers_status.size()) * j + j));
    }
    key.set_full_hash(seed);
    return seed;
}
\end{lstlisting}

Los operadores \lstinline{operator==} necesarios se implementan utilizando
los \lstinline{operator()} correspondientes.

\begin{notebox}
    Las funciones hash utilizadas en los \lstinline{operator()}
    son resistentes a colisiones,
    esto es, $
    h(State_a) \ne h(State_b) \iff State_a \ne State_b
    $
    por lo que se pueden
    utilizar para comparar elementos en \lstinline{operator==}.
\end{notebox}

\subsection{A*}

\begin{figure}
\begin{center}
\begin{tikzpicture}[node distance=2cm]
    \node (A) [startstop] {Inicio};

    \node (B) [process, right of=A, xshift=2cm] {
        Inicializar \lstinline{g_costs}, \lstinline{f_costs} y \lstinline{open_set}
    };

    \node (C) [decision, below of=B, yshift=-2cm] {
        ¿Se ha encontrado el objetivo?
    };

    \node (D) [startstop, right of=C, xshift=3.5cm] {
        Retornar
    };

    \node (E) [process, left of=C, xshift=-3.5cm] {
        Asignar el primer elemento de \lstinline{open_set}
        al nodo actual
    };

    \node (F) [process, below of=E, yshift=-1cm] {
        Calcular nodos vecinos del nodo actual
    };

    \node (G) [decision, below of=F, yshift=-2cm] {
        ¿Hay vecinos? 
    };
    
    \node (H) [process, below of=G, yshift=-2cm] {
        Calcular el coste G del vecino actual
    };

    \node (I) [decision, right of=H, xshift=3.5cm] {
        ¿Es el mejor coste G que se conoce para este estado?
    };

    \node (J) [process, right of=I, xshift=3.5cm] {
        Guardar coste G y nodo en \lstinline{g_costs} y \lstinline{f_costs}
    };

    \node (K) [decision, above of=J, yshift=2cm] {
        ¿Existe el nodo en el \lstinline{open_set}?
    };

    \node (L) [process, left of=K, xshift=-3.5cm] {
        Insertar en \lstinline{open_set}
    };

    \draw [arrow] (A) -- (B);
    \draw [arrow] (B) -- (C);
    \draw [arrow] (C) -- node [anchor=north] {Sí} (D);
    \draw [arrow] (C) -- node [anchor=north] {No} (E);
    \draw [arrow] (E) -- (F);
    \draw [arrow] (F) -- (G);
    \draw [arrow] (G) -- node [anchor=north west] {No} (C);
    \draw [arrow] (G) -- node [anchor=east] {Sí} (H);
    \draw [arrow] (H) -- (I);
    \draw [arrow] (I) -- node [anchor=north east] {No} (G);
    \draw [arrow] (I) -- node [anchor=north] {Sí} (J);
    \draw [arrow] (J) -- (K);
    \draw [arrow] (K) -- node [anchor=north east] {No} (C);
    \draw [arrow] (K) -- node [anchor=north] {Sí} (L);
    \draw [arrow] (L) -- (G);
\end{tikzpicture}
\end{center}
\caption{Representación del algoritmo A*}
\label{fig:AlgoritmoA*}
\end{figure}

\subsection{Paralelización}

\subsubsection{First Come First Serve (FCFS) Solver}
\index{First Come First Serve (FCFS) Solver}

Sería posible desarrollar una implementación que paralelice el procesamiento de los nodos,
asignando uno a cada hilo de forma que para $N$ hilos se procesen $N$ nodos
de forma simultánea.
Esta estrategia permite que los hilos procesen los nodos a medida que entran en 
el \lstinline{open_set} (Véase Figura \ref{fig:RepresentacionFCFS}).

\begin{figure}
    \begin{center}
        \begin{tikzpicture}[node distance=2cm]
            \node (00) [process] {\lstinline{open_set}};
            \node (01) [process, right of=00, xshift=6cm] {Procesa nodo};

            \draw [arrow] (00) to [bend left] node [anchor=south] {Extrae nodo} (01);
            \draw [arrow] (01) to [bend left] node [anchor=north] {Inserta vecinos} (00);
        \end{tikzpicture}
    \end{center}
    \caption{Representación de la estrategia FCFS}
    \label{fig:RepresentacionFCFS}
\end{figure}

De cualquier forma, este diseño en particular no tiene por qué reducir el tiempo
requerido para hallar un nodo solución, simplemente tiene la oportunidad de reducirlo
en algunos casos específicos.
Esto se debe a que la única diferencia entre las versiones monohilo y multihilo
es que en la multihilo se procesan más nodos en el mismo tiempo.
(Véase Figura \ref{fig:A*Comparison})

\begin{figure}
\begin{subfigure}{.5\textwidth}
\begin{center}
\begin{tikzpicture}[node distance=1.5cm]
    \node (00) [dot, fill=green!30] {00};
    \node (01) [dot, right of=00] {01};
    \node (02) [dot, right of=01] {02};
    \node (03) [dot, right of=02] {03};

    \node (10) [dot, below of=00] {10};
    \node (11) [dot, right of=10, fill=blue!30] {11};
    \node (12) [dot, right of=11] {12};
    \node (13) [dot, right of=12] {13};

    \node (20) [dot, below of=10] {20};
    \node (21) [dot, right of=20] {21};
    \node (22) [dot, right of=21, fill=blue!30] {22};
    \node (23) [dot, right of=22] {23};

    \node (30) [dot, below of=20] {30};
    \node (31) [dot, right of=30] {31};
    \node (32) [dot, right of=31] {32};
    \node (33) [dot, right of=32, fill=yellow!30] {33};

    \draw [arrow] (00) -- (11);
    \draw [arrow] (11) -- (22);
    \draw [arrow] (22) -- (33);
\end{tikzpicture}
\subcaption{Algoritmo A* monohilo}
\end{center}
\end{subfigure}
\begin{subfigure}{.5\textwidth}
\begin{center}
\begin{tikzpicture}[node distance=1.5cm]
    \node (00) [dot, fill=green!30] {00};
    \node (01) [dot, right of=00, fill=blue!30] {01};
    \node (02) [dot, right of=01] {02};
    \node (03) [dot, right of=02] {03};

    \node (10) [dot, below of=00, fill=blue!30] {10};
    \node (11) [dot, right of=10, fill=blue!30] {11};
    \node (12) [dot, right of=11, fill=blue!30] {12};
    \node (13) [dot, right of=12] {13};

    \node (20) [dot, below of=10] {20};
    \node (21) [dot, right of=20, fill=blue!30] {21};
    \node (22) [dot, right of=21, fill=blue!30] {22};
    \node (23) [dot, right of=22, fill=blue!30] {23};

    \node (30) [dot, below of=20] {30};
    \node (31) [dot, right of=30] {31};
    \node (32) [dot, right of=31, fill=blue!30] {32};
    \node (33) [dot, right of=32, fill=yellow!30] {33};

    \draw [arrow] (00) -- (11);
    \draw [arrow] (00) -- (01);
    \draw [arrow] (00) -- (10);
    \draw [arrow] (11) -- (22);
    \draw [arrow] (11) -- (12);
    \draw [arrow] (11) -- (21);
    \draw [arrow] (22) -- (33);
    \draw [arrow] (22) -- (23);
    \draw [arrow] (22) -- (32);
\end{tikzpicture}
\subcaption{Algoritmo A* multihilo (3 hilos)}
\end{center}
\end{subfigure}
\caption{Comparativa entre algoritmos monohilo y multihilo}
\label{fig:A*Comparison}
\end{figure}

\begin{notebox}
    Nótese que este acercamiento no tiene sincronización entre diferentes iteraciones,
    esto implica que la solución está sujeta a una condición de carrera
    (i.e. la solución depende de qué hilo finalice primero su ejecución).
    Por lo que sería posible ejecutar $N$ veces este algoritmo y obtener
    $N$ soluciones diferentes.
\end{notebox}

\paragraph{Secciones críticas}~

El diseño paralelo propuesto no es sin inconvenientes,
su implementación contiene varias secciones críticas que suponen
una amenaza para el rendimiento del algoritmo.
A continuación se observan cada una de estas secciones y se
analizan las razones por las cuales son necesarias.

\subparagraph{Variables de control de flujo}~

Primero, al paralelizar el algoritmo siguiendo esta estrategia,
se han añadido variables de control compartidas por todos
los hilos que sirven para conocer si se ha resuelto el problema
o no (una variable donde se copia el resultado y
otra que sirve como \italic{flag}).
El acceso a estas variables debe estar controlado
para evitar el acceso simultaneo a las mismas.
De cualquier forma, es improbable que dos hilos tengan
la necesidad de acceder esta sección crítica ya que sólo
se ejecuta una vez por lo que los efectos en el rendimiento serán nulos.
Sería necesario que dos hilos hallasen dos soluciones diferentes al
problema al mismo tiempo.

\subparagraph{\lstinline{open_set}}~

Segundo, el acceso al \lstinline{open_set} también 
debe estar controlado de forma que sólo un hilo
pueda interactuar con la estructura de datos compartida.
Esta interacción se presenta al menos en dos instancias por
cada iteración del bucle principal:
una primera vez para acceder al nodo a procesar
y otra para insertar los nuevos vecinos.
Si bien la obtención del nodo a procesar se realiza en $O(1)$
ya que el \lstinline{open_set} está ordenado y
siempre se accede al nodo en la cabeza de la lista,
la inserción de vecinos no corre la misma suerte.
Para que la lectura del nodo a procesar sea en $O(1)$
el \lstinline{open_set} se mantiene ordenado,
esto implica que la inserción ser haría en $O(n)$
\footnote{Esta implementación utiliza iteradores y
\lstinline{std::deque<T>} para hallar la posición de cada
nuevo elemento e insertarlo en el mismo barrido.}.

\begin{notebox}
    La complejidad de la sección crítica en la que se insertan
    elementos en el \lstinline{open_set}
    tiene como factor la longitud del \lstinline{open_set}. 
    Esto es, a mayor tamaño tenga el \lstinline{open_set},
    mayor tiempo será necesario para resolver la sección crítica.\\

    Nótese que a medida que avanza el programa,
    el tamaño del \lstinline{open_set} crece,
    incrementando la duración de la sección crítica y
    reduciendo la paralelización del algoritmo. 
\end{notebox}

\subsubsection{Batch Solver}
\index{Batch Solver}

La siguiente estrategia es una evolución del FCFS Solver anterior,
utiliza el mismo principio (los hilos exploran nodos del \lstinline{open_set}
a medida que éste se va llenando),
pero en este caso se implanta una barrera de sincronización en cada iteración.
Esta barrera obliga a los hilos a esperar al resto de sus compañeros
antes de extraer otro nodo del \lstinline{open_set}.

La diferencia más notable entre este acercamiento y el anterior es que
las secciones críticas se ven reducidas porque las secciones paralelas
son menores.
Además, al sincronizar los hilos en cada iteración, ahora no existe ninguna
condición de carrera que pueda alterar el resultado, por lo que
para el mismo problema este algoritmo siempre retornará el mismo resultado
y utilizará la misma ruta para llegar a él.

Se utiliza un \lstinline{std::vector<State>} para almacenar los nodos
a explorar por cada hilo y un \lstinline{std::vector<std::vector<State>>}
para que cada hilo almacene los vecinos que ha encontrado.
Cada uno de estos vectores tiene una longitud igual al número de hilos
de forma que el hilo con ID $N$ tiene asignada la posición $N$ de cada
vector.

\begin{figure}
    \begin{center}
        \begin{tikzpicture}[node distance=2cm]
            \node (00) [process] {\lstinline{open_set}};
            \node (01) [process, below of=00] {Nodos a explorar};
            \node (02) [process, below of=01] {Procesa nodo};
            \node (03) [process, below of=02] {Vecinos a insertar};

            \draw [arrow] (00) -- (01);
            
            \draw [arrow] (01.south) -- ([xshift=-15mm]00.south |- 02.north);
            \draw [arrow] (01.south) -- ([xshift=-10mm]00.south |- 02.north);
            \draw [arrow] (01.south) -- ([xshift=-5mm]00.south |- 02.north);
            \draw [arrow] (01.south) -- (00.south |- 02.north);
            \draw [arrow] (01.south) -- ([xshift=15mm]00.south |- 02.north);
            \draw [arrow] (01.south) -- ([xshift=10mm]00.south |- 02.north);
            \draw [arrow] (01.south) -- ([xshift=5mm]00.south |- 02.north);

            \draw [arrow] (02.south) -- (03);
            \draw [arrow] (02.south)+(5mm,0) -- (03);
            \draw [arrow] (02.south)+(-5mm,0) -- (03);
            \draw [arrow] (02.south)+(10mm,0) -- (03);
            \draw [arrow] (02.south)+(-10mm,0) -- (03);
            \draw [arrow] (02.south)+(15mm,0) -- (03);
            \draw [arrow] (02.south)+(-15mm,0) -- (03);

            \draw [arrow] (03) to [bend left=3cm] node [anchor=east] {Inserta vecinos} (00);
        \end{tikzpicture}
    \end{center}
    \caption{Representación de la estrategia Batch}
    \label{fig:RepresentacionBatch}
\end{figure}

\paragraph{Secciones críticas}~

A diferencia de la estrategia FCFS,
no es posible que dos hilos accedan al mismo recurso
de forma simultánea.
Las únicas estructuras de datos compartidas
(los dos \lstinline{std::vector})
ofrecen a los hilos un índice privado al que acceder.

Las secciones críticas de la estrategia FCFS
ahora son ejecutadas por un hilo:
una al registrar los nodos a explorar y
otra al retirar los vecinos e insertarlos en el \lstinline{open_set}.

\subsubsection{Recursive Solver}
\index{Recursive Solver}

La estrategia propuesta en \cite{Zag17} se basa en la ejecución
simultánea de varias instancias del algoritmo A* en el mismo problema.
\begin{enumerate}[itemsep=0.25px]
    \item Calcular vecinos del estado inicial.
    \item Asignar un hilo a cada vecino.
    \item Para cada vecino, resolver el problema como si fuese el estado inicial.
    \item Recoger resultados obtenidos.
    \item Obtener resultado con menor coste.
    \item Retornar.
\end{enumerate}

Al igual que la solución por batches,
no existe ninguna condición de carrera que permita
al algoritmo retornar resultados diferentes
en varias ejecuciones.
Originalmente cada hilo utiliza sus propias estructuras
de datos, por lo que no existen secciones críticas.
No obstante, sería posible hacer una versión en la que
los distintos hilos compartiesen las estructuras de costes
y el \lstinline{open_set}.
Implementar el algoritmo de esta forma añadiría las
tediosas secciones críticas que podrían ser más
dañinas que beneficiosas.

Este diseño puede ser de gran interés para otros problemas
distintos al JSP donde existan varios estados iniciales,
ya que sería posible calcular una solución del A*
para cada uno de ellos.
El algoritmo permitiría conocer el mejor estado inicial
así como la ruta a seguir para llegar al estado objetivo.

\subsection{Hash Distributed A* (HDA*) Solver}
\index{Hash Distributed A* (HDA*) Solver}

El algoritmo propuesto en \cite{KFB09}
utiliza tantos \lstinline{open_set} como hilos
y una función hash para asignar cada nodo a uno de los
\lstinline{open_set}.
Cada hilo es `propietario' de uno de los \lstinline{open_set}
y por consecuente, de los nodos que estén contenidos
dentro del mismo.
Cada hilo está encargado de explorar los nodos de su
\lstinline{open_set} y de añadir sus vecinos
al \lstinline{open_set} correspondiente.

\Chapter{Experimentos}{Trabajo y Resultados}

\section{Conjuntos de datos}

Los diferentes conjuntos de datos utilizados para medir el rendimiento
de las diferentes implementaciones han sido obtenidos de
\href{http://jobshop.jjvh.nl/}{(HTTP) Jobshop Instances}
o diseñados a mano.

\subsection{\italic{Jobshop Instances}}

\subsection{Personalizados}

\section{Método de medición}

Todas las versiones imprimen por salida estándar datos que posteriormente son
procesados en formato CSV:

\begin{itemize}[itemsep=0.25px]
    \item Lenguaje
    \item Número de hilos
    \item Porcentaje del trabajo resuelto
    \item Trabajos
    \item Tareas
    \item Trabajadores
    \item Tiempo de ejecución
    \item Planificación
    \item \italic{Makespan}
\end{itemize}

La alta complejidad del JSP implica que un mínimo aumento en el tamaño del problema
puede implicar que el algoritmo no finalice su ejecución en un tiempo polinomial.
Por ello, en lugar de tomar una única medición al inicio y final de la ejecución
se ha optado por utilizar un objeto personalizado que toma mediciones en intervalos
predefinidos.
De esta forma, de una única ejecución se podría obtener:

\begin{itemize}[itemsep=0.25px]
    \item Tiempo de inicio.
    \item Tiempo necesario para resolver el $10\percentsign$ del problema.
    \item Tiempo necesario para resolver el $20\percentsign$ del problema.
    \item \dots
    \item Tiempo necesario para resolver el $90\percentsign$ del problema.
    \item Tiempo necesario para resolver el $100\percentsign$ del problema.
\end{itemize}
Se utiliza una función que se encarga de tomar una medición de tiempo
antes y después de iniciar el algoritmo.

\begin{lstlisting}[language=C++]
class Chronometer
{
private:
    std::chrono::_V2::system_clock::time_point m_start_time;
    std::map<unsigned short, bool> m_goals;
    std::map<unsigned short, double> m_times;
    std::string m_solver_name;

public:
    Chronometer() : Chronometer(
        std::map<unsigned short, bool>(),
        "Unknown Solver") {}
    explicit Chronometer(
        const std::map<unsigned short, bool> &goals,
        std::string const &solver_name) : m_goals(goals),
                                          m_solver_name(solver_name) {}

    void start()
    {
        this->m_start_time = std::chrono::high_resolution_clock::now();
    }
    std::chrono::duration<double> time() const
    {
        return (
            std::chrono::high_resolution_clock::now() -
            this->m_start_time
        );
    }

    void process_iteration(const State &state);
    void log_timestamp(unsigned short goal, const State &state);
    void enable_goals();
    std::map<unsigned short, double> get_timestamps() const;
};
\end{lstlisting}

Esta clase se utiliza para tomar las mediciones de todas las iteraciones
\footnote{Ignorando la primera para calentar la caché}
y posteriormente estos datos se utilizan para calcular tiempos medios de ejecución,
máximos y mínimos.

\subsection{FPGA}

\section{Resultados}

\section{Análisis}

\Chapter{Conclusiones}{Observaciones y Trabajos futuros}

%----------------------------------------------------------------------------------------
%	THESIS CONTENT - APPENDICES
%----------------------------------------------------------------------------------------

\appendix % Cue to tell LaTeX that the following "chapters" are Appendices

% Include the appendices of the thesis as separate files from the Appendices folder
% Uncomment the lines as you write the Appendices

%\include{Appendices/AppendixA}
\Chapter{Código Fuente}{}

\section{Chronometer}

\begin{lstlisting}[language=C++, caption=Clase Chronometer utilizada para tomar mediciones, label=lst:Chronometer]
class Chronometer
{
private:
    std::chrono::_V2::system_clock::time_point m_start_time;
    std::map<unsigned short, bool> m_goals;
    std::map<unsigned short, double> m_times;
    std::string m_solver_name;

public:
    Chronometer() : Chronometer(
        std::map<unsigned short, bool>(),
        "Unknown Solver") {}
    explicit Chronometer(
        const std::map<unsigned short, bool> &goals,
        std::string const &solver_name) : m_goals(goals),
                                          m_solver_name(solver_name) {}

    void start()
    {
        this->m_start_time = std::chrono::high_resolution_clock::now();
    }
    std::chrono::duration<double> time() const
    {
        return (
            std::chrono::high_resolution_clock::now() -
            this->m_start_time
        );
    }

    void process_iteration(const State &state);
    void log_timestamp(unsigned short goal, const State &state);
    void enable_goals();
    std::map<unsigned short, double> get_timestamps() const;
};
\end{lstlisting}

\section{Random Solver}
\label{sec:RandomSolver}

\begin{lstlisting}[language=C++, caption=Clase RandomSolver utilizada para resolver un problema aleatoriamente]
class RandomSolver : public AStarSolver
{
private:
    State get_random_next_state(const State &, std::mt19937 &) const;

public:
    RandomSolver() = default;
    ~RandomSolver() override = default;

    State solve(
        std::vector<std::vector<Task>>,
        std::size_t,
        Chronometer &) const override;

    std::string get_name() const override { return "RANDOM Solver"; };
};

State RandomSolver::solve(
    std::vector<std::vector<Task>> jobs,
    std::size_t n_workers,
    Chronometer &c) const
{
    const std::size_t nJobs = jobs.size();
    if (nJobs == 0)
        return State();
    const std::size_t nTasks = jobs[0].size();
    if (nTasks == 0)
        return State();
    if (n_workers == 0)
        n_workers = nTasks;

    std::random_device rd;
    std::mt19937 gen(rd());

    const auto startingSchedule = std::vector<std::vector<int>>(nJobs, std::vector<int>(nTasks, -1));
    const auto startingWorkersStatus = std::vector<int>(n_workers, 0);
    auto currentState = State(jobs, startingSchedule, startingWorkersStatus);

    while (!currentState.is_goal_state())
    {
        currentState = this->get_random_next_state(currentState, gen);
        c.process_iteration(currentState);
    }

    return currentState;
}

State RandomSolver::get_random_next_state(const State &currentState, std::mt19937 &gen) const
{
    std::vector<State> neighbors = currentState.get_neighbors_of();
    std::uniform_int_distribution<> dis(0, (int)neighbors.size() - 1);
    return neighbors[dis(gen)];
}
\end{lstlisting}

\section{Read and Cut}
\label{sec:ReadAndCut}

\begin{lstlisting}[
    language=C++,
    caption=Funciones utilizadas para leer y cortar conjuntos de datos
]
struct ReadTaskStruct
{
    unsigned int duration;
    std::vector<unsigned int> qualified_workers;
};

std::vector<std::vector<Task>> get_jobs_from_file(
    const std::string &filename
) {
    std::ifstream file(filename);
    std::string my_string;
    std::string my_number;
    std::vector<std::vector<ReadTaskStruct>> data;
    std::vector<std::vector<Task>> out;
    if (!file.is_open())
    {
        std::cerr << "Could not read file" << std::endl;
        file.close();
        return out;
    }
    while (std::getline(file, my_string))
    {
        std::stringstream line(my_string);
        data.emplace_back();
        bool is_worker = true;
        unsigned int qualified_worker = 0;
        while (std::getline(line, my_number, ';'))
        {
            if (is_worker)
            {
                qualified_worker = stoi(my_number);
                is_worker = false;
            }
            else
            {
                data[data.size() - 1].emplace_back();
                data[data.size() - 1][
                    data[data.size() - 1].size() - 1
                ].duration = stoi(my_number);
                data[data.size() - 1][
                    data[data.size() - 1].size() - 1
                ].qualified_workers = std::vector<unsigned int>(
                    1,
                    qualified_worker
                );
                is_worker = true;
            }
        }
    }
    for (std::size_t job_idx = 0; job_idx < data.size(); job_idx++)
    {
        out.emplace_back();
        const std::vector<struct ReadTaskStruct> currentJob = data[
            job_idx
        ];
        for (
            std::size_t task_idx = 0;
            task_idx < currentJob.size();
            task_idx++
        ) {
            const struct ReadTaskStruct currentTask = currentJob[
                task_idx
            ];
            unsigned int h_cost = 0;
            for (
                std::size_t unscheduled_task_idx = task_idx;
                unscheduled_task_idx < currentJob.size();
                unscheduled_task_idx++
            )
                h_cost += currentJob[unscheduled_task_idx].duration;
            out[job_idx].emplace_back(
                currentTask.duration,
                h_cost,
                currentTask.qualified_workers
            );
        }
    }
    file.close();
    return out;
}

std::vector<std::vector<Task>> cut(
    std::vector<std::vector<Task>> jobs,
    float percentage
) {
    percentage = percentage < 0 ? 0 : percentage;
    percentage = percentage > 1 ? 1 : percentage;
    const unsigned int jobs_to_keep = int(
        float(jobs.size()) * percentage
    );
    const unsigned int tasks_to_keep = int(
        float(jobs[0].size()) * percentage
    );
    std::vector<std::vector<Task>> new_jobs;

    for (unsigned int job_idx = 0; job_idx < jobs_to_keep; job_idx++)
    {
        new_jobs.emplace_back();
        for (
            unsigned int task_idx = 0;
            task_idx < tasks_to_keep;
            task_idx++
        )
            new_jobs[job_idx].push_back(jobs[job_idx][task_idx]);
    }

    return new_jobs;
}
\end{lstlisting}

\pagebreak
\section{Heurísticos}

\subsection{Slow - Óptimo}
\label{ssec:SlowOptimo}

\begin{lstlisting}[language=C++]
unsigned int State::calculate_h_cost() const
{
    std::vector<int> h_costs;
    for (size_t job_idx = 0; job_idx < this->m_jobs.size(); job_idx++)
    {
        h_costs.emplace_back(0);
        std::vector<Task> job = this->m_jobs[job_idx];
        for (size_t task_idx = 0; task_idx < job.size(); task_idx++)
            if (this->get_schedule()[job_idx][task_idx] == -1)
                h_costs[job_idx] += job[task_idx].get_duration();
    }
    auto max_element = std::max_element(h_costs.begin(), h_costs.end());
    return max_element == h_costs.end() ? 0 : *max_element;
}
\end{lstlisting}

\subsection{Fast}
\label{ssec:Fast}

\begin{lstlisting}[language=C++]
unsigned int State::calculate_h_cost() const
{
    unsigned int unscheduled_tasks_count = 0;
    for (std::size_t job_idx = 0; job_idx < this->m_jobs.size(); job_idx++)
        for (std::size_t task_idx = 0; task_idx < this->m_jobs[job_idx].size(); task_idx++)
            if (this->m_schedule[job_idx][task_idx] == -1)
                unscheduled_tasks_count += this->m_jobs[job_idx][task_idx].get_duration();
    return unscheduled_tasks_count; 
}
\end{lstlisting}

\pagebreak

\Chapter{Resultados y Métricas}{}

%----------------------------------------------------------------------------------------
%	BIBLIOGRAPHY
%----------------------------------------------------------------------------------------

\nocite{*}
\printbibliography[heading=bibintoc]

%----------------------------------------------------------------------------------------

\addchaptertocentry{Índice alfabético}
\printindex

\end{document}  
