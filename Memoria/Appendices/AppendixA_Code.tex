\Chapter{Código Fuente}{}

\section{Chronometer}

\begin{lstlisting}[language=C++, caption=Clase Chronometer utilizada para tomar mediciones, label=lst:Chronometer]
class Chronometer
{
private:
    std::chrono::_V2::system_clock::time_point m_start_time;
    std::map<unsigned short, bool> m_goals;
    std::map<unsigned short, double> m_times;
    std::string m_solver_name;

public:
    Chronometer() : Chronometer(
        std::map<unsigned short, bool>(),
        "Unknown Solver") {}
    explicit Chronometer(
        const std::map<unsigned short, bool> &goals,
        std::string const &solver_name) : m_goals(goals),
                                          m_solver_name(solver_name) {}

    void start()
    {
        this->m_start_time = std::chrono::high_resolution_clock::now();
    }
    std::chrono::duration<double> time() const
    {
        return (
            std::chrono::high_resolution_clock::now() -
            this->m_start_time
        );
    }

    void process_iteration(const State &state);
    void log_timestamp(unsigned short goal, const State &state);
    void enable_goals();
    std::map<unsigned short, double> get_timestamps() const;
};
\end{lstlisting}

\section{Random Solver}
\label{sec:RandomSolver}

\begin{lstlisting}[language=C++, caption=Clase RandomSolver utilizada para resolver un problema aleatoriamente]
class RandomSolver : public AStarSolver
{
private:
    State get_random_next_state(const State &, std::mt19937 &) const;

public:
    RandomSolver() = default;
    ~RandomSolver() override = default;

    State solve(
        std::vector<std::vector<Task>>,
        std::size_t,
        Chronometer &) const override;

    std::string get_name() const override { return "RANDOM Solver"; };
};

State RandomSolver::solve(
    std::vector<std::vector<Task>> jobs,
    std::size_t n_workers,
    Chronometer &c) const
{
    const std::size_t nJobs = jobs.size();
    if (nJobs == 0)
        return State();
    const std::size_t nTasks = jobs[0].size();
    if (nTasks == 0)
        return State();
    if (n_workers == 0)
        n_workers = nTasks;

    std::random_device rd;
    std::mt19937 gen(rd());

    const auto startingSchedule = std::vector<std::vector<int>>(nJobs, std::vector<int>(nTasks, -1));
    const auto startingWorkersStatus = std::vector<int>(n_workers, 0);
    auto currentState = State(jobs, startingSchedule, startingWorkersStatus);

    while (!currentState.is_goal_state())
    {
        currentState = this->get_random_next_state(currentState, gen);
        c.process_iteration(currentState);
    }

    return currentState;
}

State RandomSolver::get_random_next_state(const State &currentState, std::mt19937 &gen) const
{
    std::vector<State> neighbors = currentState.get_neighbors_of();
    std::uniform_int_distribution<> dis(0, (int)neighbors.size() - 1);
    return neighbors[dis(gen)];
}
\end{lstlisting}

\section{Read and Cut}
\label{sec:ReadAndCut}

\begin{lstlisting}[
    language=C++,
    caption=Funciones utilizadas para leer y cortar conjuntos de datos
]
struct ReadTaskStruct
{
    unsigned int duration;
    std::vector<unsigned int> qualified_workers;
};

std::vector<std::vector<Task>> get_jobs_from_file(
    const std::string &filename
) {
    std::ifstream file(filename);
    std::string my_string;
    std::string my_number;
    std::vector<std::vector<ReadTaskStruct>> data;
    std::vector<std::vector<Task>> out;
    if (!file.is_open())
    {
        std::cerr << "Could not read file" << std::endl;
        file.close();
        return out;
    }
    while (std::getline(file, my_string))
    {
        std::stringstream line(my_string);
        data.emplace_back();
        bool is_worker = true;
        unsigned int qualified_worker = 0;
        while (std::getline(line, my_number, ';'))
        {
            if (is_worker)
            {
                qualified_worker = stoi(my_number);
                is_worker = false;
            }
            else
            {
                data[data.size() - 1].emplace_back();
                data[data.size() - 1][
                    data[data.size() - 1].size() - 1
                ].duration = stoi(my_number);
                data[data.size() - 1][
                    data[data.size() - 1].size() - 1
                ].qualified_workers = std::vector<unsigned int>(
                    1,
                    qualified_worker
                );
                is_worker = true;
            }
        }
    }
    for (std::size_t job_idx = 0; job_idx < data.size(); job_idx++)
    {
        out.emplace_back();
        const std::vector<struct ReadTaskStruct> currentJob = data[
            job_idx
        ];
        for (
            std::size_t task_idx = 0;
            task_idx < currentJob.size();
            task_idx++
        ) {
            const struct ReadTaskStruct currentTask = currentJob[
                task_idx
            ];
            unsigned int h_cost = 0;
            for (
                std::size_t unscheduled_task_idx = task_idx;
                unscheduled_task_idx < currentJob.size();
                unscheduled_task_idx++
            )
                h_cost += currentJob[unscheduled_task_idx].duration;
            out[job_idx].emplace_back(
                currentTask.duration,
                h_cost,
                currentTask.qualified_workers
            );
        }
    }
    file.close();
    return out;
}

std::vector<std::vector<Task>> cut(
    std::vector<std::vector<Task>> jobs,
    float percentage
) {
    percentage = percentage < 0 ? 0 : percentage;
    percentage = percentage > 1 ? 1 : percentage;
    const unsigned int jobs_to_keep = int(
        float(jobs.size()) * percentage
    );
    const unsigned int tasks_to_keep = int(
        float(jobs[0].size()) * percentage
    );
    std::vector<std::vector<Task>> new_jobs;

    for (unsigned int job_idx = 0; job_idx < jobs_to_keep; job_idx++)
    {
        new_jobs.emplace_back();
        for (
            unsigned int task_idx = 0;
            task_idx < tasks_to_keep;
            task_idx++
        )
            new_jobs[job_idx].push_back(jobs[job_idx][task_idx]);
    }

    return new_jobs;
}
\end{lstlisting}

\pagebreak
\section{Heurísticos}

\subsection{Slow - Óptimo}
\label{ssec:SlowOptimo}

\begin{lstlisting}[language=C++]
unsigned int State::calculate_h_cost() const
{
    std::vector<int> h_costs;
    for (size_t job_idx = 0; job_idx < this->m_jobs.size(); job_idx++)
    {
        h_costs.emplace_back(0);
        std::vector<Task> job = this->m_jobs[job_idx];
        for (size_t task_idx = 0; task_idx < job.size(); task_idx++)
            if (this->get_schedule()[job_idx][task_idx] == -1)
                h_costs[job_idx] += job[task_idx].get_duration();
    }
    auto max_element = std::max_element(h_costs.begin(), h_costs.end());
    return max_element == h_costs.end() ? 0 : *max_element;
}
\end{lstlisting}

\subsection{Fast}
\label{ssec:Fast}

\begin{lstlisting}[language=C++]
unsigned int State::calculate_h_cost() const
{
    unsigned int unscheduled_tasks_count = 0;
    for (std::size_t job_idx = 0; job_idx < this->m_jobs.size(); job_idx++)
        for (std::size_t task_idx = 0; task_idx < this->m_jobs[job_idx].size(); task_idx++)
            if (this->m_schedule[job_idx][task_idx] == -1)
                unscheduled_tasks_count += this->m_jobs[job_idx][task_idx].get_duration();
    return unscheduled_tasks_count; 
}
\end{lstlisting}

\pagebreak
