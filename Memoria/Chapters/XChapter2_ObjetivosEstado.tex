\Chapter{Objetivos y estado actual}{Objetivos y estado del arte}

\section{Objetivos}
\index{Objetivos y estado actual!Objetivos}

Este proyecto tiene como objetivos principales el estudio y análisis
del algoritmo A* enfocado a solucionar el Job Shop Scheduling Problem.
Se llevará a cabo la implementación y
optimización del mismo junto con el desarrollo y
comparación de distintas soluciones paralelas.
Adicionalmente, se analiza el funcionamiento de una FPGA y
las posibilidades que tiene este dispositivo de incrementar
el rendimiento del algoritmo.

En detalle, este proyecto contiene:
\begin{itemize}[itemsep=0.25px]
    \item El Algoritmo A*: Funcionamiento monohilo del algortimo A* utilizado para resolver el JSP.
    \item Implementación de A*: Implementación monohilo del algoritmo A*.
    \item Optimización monohilo: Análisis y optimización de cuellos de botella.
    \item Paralelización: Análisis de distintas alternativas para paralelizar el algoritmo.
    \item FPGA: Funcionamiento y oportunidades de una FPGA.
    \item Heurísticos: Comparativa empírica de diferentes funciones heurísticas.
    \item Dijkstra: Comparativa empírica con el algoritmo de Dijkstra.
    \item Comparativa de algoritmos paralelos: Observaciones empíricas del rendimiento de los distintos algoritmos.
\end{itemize}

\pagebreak

\section{Estado del arte}
\index{Objetivos y estado actual!Estado del arte}

Este proyecto abarca diversos tópicos, cuyas bibliografías
(incluso tomadas de una en una) tienen una gran extensión.
A pesar de ello, resulta complicado hallar estudios previos sobre implementaciones
paralelas del algoritmo A* enfocadas a la resolución del Job Shop Scheduling Problem
utilizando FPGAs.
Así pues, el punto de partida de este proyecto
se compone principalmente de estudios
sobre los distintos tópicos de forma individual.

El Job Shop Scheduling Problem tiene su origen en la década de 1960,
desde entonces ha sido utilizado frecuentemente (incluso hasta el día de hoy)
como herramienta de medición del rendimiento de algoritmos que sean 
capaces de resolverlo \cite{Man67}.

A lo largo de los años, se han realizado numerosos trabajos con el objetivo
de recoger distintos algoritmos que resuelvan el problema.
Dichos algoritmos provienen de diferentes ámbitos de la computación.
Entre ellos se pueden encontrar búsquedas en grafos,
listas de prioridad, ramificación y poda, algoritmos genéticos,
simulaciones Monte Carlo o
métodos gráficos como diagramas Gantt y Pert.
\cite{Yan77}, \cite{Nil69}, \cite{KTM99}, \cite{BC22}, \cite{Pin08}.

El algoritmo A* fue diseñado a finales de la década de 1960
con el objetivo de implementar el enrutamiento de un robot
conocido como \italic{"Shakey the Robot"} \cite{Nil84}.
A* es una evolución del conocido algoritmo de Dijkstra
frecuentemente utilizado también para la búsqueda en grafos.
La principal diferencia entre estos dos algoritmos es el
uso de una función heurística en el A* que `guía' al algoritmo
en la dirección de la solución.
\cite{HNR68}, \cite{MSV13}, \cite{Kon14}.

El algoritmo A* no es dado a la paralelización,
no existe una implementación simple que aproveche el funcionamiento de
varios procesadores de forma simultánea.
En su lugar existen varias alternativas que paralelizan el algoritmo,
cada una de ellas con sus fortalezas y debilidades.
Estas diferentes versiones serán estudiadas, implementadas y
probadas en esta investigación.
\cite{Zag17}, \cite{WH16}.

El tópico sobre el que menor cantidad de documentación existe
y que supone una mayor curva de aprendizaje es sin lugar a duda
la implementación del algoritmo A* en una FPGA.
La principal conclusión de la literatura en este área es
que el uso del hardware incrementa el rendimiento del
algoritmo y reduce su coste energético al mismo tiempo.
En función de las herramientas utilizadas para implementar
el algoritmo las ganancias son distintas,
estudios en los que se utilizaron sintetizadores (HLS)
para generar el código de la FPGA obtuvieron resultados
menos interesantes que aquellos que diseñaron en hardware
directamente.
\cite{ZJW20}, \cite{NSG17}.

\pagebreak

\section{Job Shop Scheduling Problem}
\index{Objetivos y estado actual!Job Shop Scheduling Problem}

Se estudia la implementación de una solución al problema
\italic{Job Shop Scheduling (JSP)}~\cite{Yan77}
\footnote{El problema también es conocido por otros nombres similares
como \italic{Job Shop Scheduling Problem (JSSP)} o Job Scheduling Problem (JSP).}
utilizando el algoritmo A*.
Como su nombre indica, se trata de un problema en el que se debe
crear una planificación.
Como se especifica en la literatura, el JSP es un problema de optimización np-hard.

El JSP busca una planificación para una serie de
máquinas (o trabajadores) que deben realizar un número conocido
de trabajos.
Cada trabajo está formado por una serie de operaciones (o tareas),
con una duración conocida.
Las tareas de un mismo trabajo deben ser ejecutadas en un orden específico.

Existen numerosas variantes de este problema,
algunas permiten la ejecución
en paralelo de algunas tareas o requieren que alguna tarea
en específico sea ejecutada por un trabajador (o tipo de trabajador)
en particular.
Por ello, es clave denotar las restricciones que definen el problema JSP\@:

\begin{enumerate}
    \item Existen un número natural conocido de trabajos.
    \item A excepción de la primera tarea de cada trabajo,
    todas tienen una única tarea predecesora del mismo trabajo
    que debe ser completada antes de iniciar su ejecución.
    \item Cada tarea puede tener una duración distinta.
    \item La duración de cada tarea es un número natural conocido.
    \item Existe un número natural conocido de trabajadores.
    \item Cada tarea tiene un trabajador asignado,
    de forma que sólo ese trabajador puede ejecutar la tarea.
    \item Una vez iniciada una tarea, no se puede interrumpir su ejecución.
    \item Un mismo trabajador puede intercalar la ejecución de tareas de diferentes trabajos.
    \item Un trabajador sólo puede realizar una tarea al mismo tiempo.
    \item Los tiempos de preparación de un trabajador antes de realizar una tarea son nulos.
    \item Los tiempos de espera entre la realización de una tarea y otra son nulos.
\end{enumerate}

\begin{keynotebox}
    El Job Shop Scheduling Problem (JSP) es un problema np-hard que
    consiste en crear una planificación
    para ejecutar una serie de tareas que tienen una duración y están asignadas
    a un trabajador en particular.
\end{keynotebox}

\subsection{Notación}

A continuación se describe una notación utilizada en los ejemplos
de este documento y en diversas explicaciones.

El problema del JSP consiste en la planificación de un número natural ($M$)
de trabajos $(J_0 \dots J_{M-1})$ donde cada trabajo $J_i$
está formado por un número ($N$) de tareas $(T_{i,0} \dots T_{i,N-1})$.

A cada tarea $T_{i,j}$ le corresponden el par de atributos ($D_{i,j}$ y $W_{i,j}$)
donde $D_{i,j}$ es la duración de la tarea
y $W_{i,j}$ es el trabajador que la debe ejecutar.

\subsubsection{Conjunto de datos}

El conjunto de datos es el inicio del problema a resolver,
contiene los datos de entrada necesarios para ejecutar cualquier
algoritmo.
En particular, está compuesto por un listado de trabajos
formado por sus tareas, duraciones y trabajadores:
\begin{enumerate}[start=0, itemsep=0.25px]
    \item (2, A), (5, B), (1, C)
    \item (3, B), (3, C), (3, A)
\end{enumerate}
En este ejemplo, el conjunto de datos estaría formado por dos trabajos
($J_0, J_1$)
con tres tareas cada uno ($T_{i,0} \dots T_{i,2}$).
Las tareas serán ejecutadas por tres trabajadores diferentes (A, B y C).

La tarea $T_{0,0}$ tiene una duración de dos instantes y
debe ser realizada por el trabajador A.
La tarea $T_{1,2}$ tiene una duración de tres instantes y
también debe ser realizada por el trabajador A.
Por lo que estas dos tareas no podrán ser ejecutadas de forma simultánea ya que ambas requieren al
mismo trabajador.

\subsubsection{Planificación}

La planificación es un dato propio del estado.
Indica el instante de tiempo en el que se inicia cada tarea
de forma que el elemento en la posición $i$ del vector se
corresponde con el instante de inicio de la tarea $i$:
$(0, 3, -1)$ indica que la tarea 0 se inicia en el instante 0,
la 1 en el 3 y la 2 aún está sin planificar.

Generalmente se muestra la planificación de todo el conjunto de datos
utilizando un listado:
\begin{enumerate}[start=0, itemsep=0.25px]
    \item $(0, 3, -1)$
    \item $(0, -1, -1)$
\end{enumerate}
La $T_{0,0}$ se inicia en el instante 0 mientras que la $T_{1,1}$ está aún sin planificar.

\subsubsection{Estado trabajadores}

El estado de los trabajadores muestra en forma de lista
el instante de tiempo donde cada trabajador queda libre.
Simbólicamente, $w_i$ corresponde al instante
en el que el trabajador $i$ queda libre y
$W$ es el número total de trabajadores.

En los ejemplos, la estructura está construida de forma que el elemento $k$
corresponde al instante de tiempo donde el trabajador $k$
queda libre.
$(2, 8, 0)$ indica que el trabajador 0 esta libre a partir del instante 2
($w_0 = 2$),
el 1 a partir del 8
($w_1 = 8$)
y el 2 a partir del 0
($w_2 = 0$).

Aunque esta información no es parte de la salida del programa,
es necesaria para el cómputo de la solución.

\subsubsection{Función objetivo, \italic{Makespan}}
\index{Makespan}

Existen varios objetivos que se pueden buscar con este problema
de optimización.
En este caso se selecciona la minimización del \italic{makespan}.

El \italic{makespan} es el tiempo necesario para completar
todos los trabajos de un conjunto de datos,
es el `resultado' que genera el algoritmo.
Lógicamente, el \italic{makespan} óptimo es el mínimo.
Algunas versiones del algoritmo desarrolladas en este proyecto
no tienen la certeza de obtener el resultado óptimo.
Por ello, si una implementación $A$ retorna
un \italic{makespan} menor que otra $B$,
$A$ es mejor que $B$. \footnote{
    Suponiendo que el resto de métricas entre $A$ y $B$
    son lo suficientemente similares.
}

Dado un estado final (con todas sus tareas planificadas)
el \italic{makespan} se define como el $cost_g$ de ese estado.
O lo que es lo mismo, el instante en el que el último
trabajador queda libre:
$$
    \max_{0 < i < W}(w_i)
$$

\begin{examplebox}
    Sea la siguiente planificación:
    \begin{itemize}[itemsep=0.25px]
        \item Planificación:
        \begin{enumerate}[itemsep=0.25px]
            \item (0, 3, 5)
            \item (0, 3, 3)
        \end{enumerate}
        \item Estado trabajadores: (2, 8, 3)
    \end{itemize}

    El \italic{makespan} es $max([2, 8, 3]) = 8$.
\end{examplebox}

\pagebreak
\subsection{Ejemplo}
\label{ssec:A*Example}

A continuación se muestra una posible solución para un ejemplo de pequeño tamaño
donde cada línea corresponde a un trabajo y cada tupla al par (duración, trabajador).
\begin{enumerate}[start=0, itemsep=0.25px]
    \item (2, A), (5, B), (1, C)
    \item (3, B), (3, C), (3, A)
\end{enumerate}

Los siguientes diagramas (Figura \ref{fig:DiagramaRandom} y \ref{fig:PlanificacionEjemplo})
representan los trabajos y planificación respectivamente.
En el primero cada nodo representa una tarea,
los arcos continuos el orden
y los discontinuos el plan de cada trabajador.
El tiempo de inicio
de cada tarea viene dado por la longitud del camino más largo
desde $i$ hasta el nodo de esa tarea,
mientras que el makespan sería el camino más largo entre $i$ y $f$.
El segundo diagrama se conoce como diagrama de Gantt y
muestra el instante de inicio, fin y duración de cada trabajo.
\footnote{
    Otros diagramas: Pert (\ref{sec:Pert}) y Gantt (\ref{sec:Gantt}).
}

\begin{figure}[h]
    \begin{center}
        \begin{tikzpicture}[node distance=2cm]
            \node (i) [dot] {i};

            \node (T00) [dot, right of=i, above of=i]  {$T_{0,0}$};
            \node (T01) [dot, right of=T00]  {$T_{0,1}$};
            \node (T02) [dot, right of=T01]  {$T_{0,2}$};

            \node (T10) [dot, right of=i, below of=i]  {$T_{1,0}$};
            \node (T11) [dot, right of=T10]  {$T_{1,1}$};
            \node (T12) [dot, right of=T11]  {$T_{1,2}$};

            \node (f) [dot, right of=T02, below of=T02] {f};

            \draw [arrow] (i.east) to node [anchor=south east] {0} (T00);
            \draw [arrow] (i.east) to node [anchor=north east] {0} (T10);

            \draw [arrow] (T00.east) to node [anchor=south] {2} (T01);
            \draw [arrow] (T10.east) to node [anchor=north] {3} (T11);

            \draw [arrow] (T01.east) to node [anchor=south west] {5} (T02);
            \draw [arrow] (T11.east) to node [anchor=north west] {3} (T12);

            \draw [arrow] (T02.east) to node [anchor=south west] {1} (f);
            \draw [arrow] (T12.east) to node [anchor=north west] {3} (f);

            \draw [dorrow,fill=yellow!60] (T00.east) to node [anchor=south west] {2} (T12);
            \draw [dorrow,fill=green!60] (T10.east) to node [anchor=south east] {3} (T01);
            \draw [dorrow,fill=red!60] (T11.east) to node [anchor=south east] {3} (T02);
        \end{tikzpicture}
    \end{center}
    \caption{Diagrama ejemplo.}
    \label{fig:DiagramaRandom}
\end{figure}
\begin{figure}[h]
    \begin{center}
        \begin{tikzpicture}[node distance=4cm]
            \draw [step=1cm,gray,very thin] (0.1, 0.1) grid (10.9, 4.9);
            \draw [arrow] (0,0) -- (11,0) node[anchor=north west] {Tiempo};
            \foreach \x in {0, 1, 2, 3, 4, 5, 6, 7, 8, 9}
                \draw (1cm+\x cm,1pt) -- (1cm+\x cm,-1pt) node[anchor=north] {$\x$};
            \draw [arrow] (0,0) -- (0,5) node[anchor=south east] {Trabajos};
            \draw (1pt,3.5cm) -- (-1pt,3.5cm) node[anchor=east] {$J_0$};
            \draw (1pt,1.5cm) -- (-1pt,1.5cm) node[anchor=east] {$J_1$};

            \node[
                rectangle,
                draw,
                fill=yellow!60,
                minimum width=2cm,
                minimum height=1cm
            ] (r) at (2,3.5) {$T_{0,0}$};
            \node[
                rectangle,
                draw,
                fill=green!60,
                minimum width=5cm,
                minimum height=1cm
            ] (r) at (6.5,3.5) {$T_{0,1}$};
            \node[
                rectangle,
                draw,
                fill=red!60,
                minimum width=1cm,
                minimum height=1cm
            ] (r) at (9.5,3.5) {$T_{0,2}$};

            \node[
                rectangle,
                draw,
                fill=green!60,
                minimum width=3cm,
                minimum height=1cm
            ] (r) at (2.5,1.5) {$T_{1,0}$};
            \node[
                rectangle,
                draw,
                fill=red!60,
                minimum width=3cm,
                minimum height=1cm
            ] (r) at (5.5,1.5) {$T_{1,1}$};
            \node[
                rectangle,
                draw,
                fill=yellow!60,
                minimum width=3cm,
                minimum height=1cm
            ] (r) at (8.5,1.5) {$T_{1,2}$};
        \end{tikzpicture}
    \end{center}
    \caption{Diagrama planificación ejemplo.}
    \label{fig:PlanificacionEjemplo}
\end{figure}

\pagebreak