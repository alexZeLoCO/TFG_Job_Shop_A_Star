\Chapter{Planificación}{}
\label{cha:Planificación}

A continuación se describe la planificación utilizada para llevar a cabo
esta investigación.
La duración de cada tarea se mide en semanas correspondientes a
5 días cada una, cada día trabajando una jornada parcial de 4 horas.

Esta investigación se ha llevado a cabo utilizando una combinación entre
SCRUM y TDD.
El estudio del algortimo A* se ha dividido en tareas de menor tamaño
que han sido resueltas de forma ordenada utilizando herramientas como
el Kanban.
Los programas se han desarrollado utilizando \italic{Test Driven Development} (TDD)
en el que primero se crean juegos de pruebas de software y
posteriormente se codifica hasta que todas las pruebas retornan los resultados esperados.

La tabla siguiente detalla las tareas realizadas así como su duración e instantes
de inicio y fin.
Todas las duraciones se han medido en semanas,
esta decisión ha facilitado la clara distinción entre el fin de una tarea y
el inicio de otra.
Así mismo, los datos de las columnas \italic{Comienzo} y \italic{Fin}
corresponden al número de semana del año 2024 en la que cada tarea ha tenido lugar.
La semana de inicio es incluida mientras que la de fin es sin incluir.
El trabajo se ha dividido en cinco fases (0-4).
A las tareas se les ha proporcionado un ID cuyo primer dígito corresponde
a la fase de la tarea.

\begin{multicols}{2}
\begin{enumerate}[start=0,itemsep=0.25px]
    \item Estudios previos.
    \item Desarrollo de prototipo Python.
    \item Desarrollo de programa C++.
    \item Síntesis de programa VHDL.
    \item Redacción de memoria.
\end{enumerate}
\end{multicols}

Se ha trabajado un total de 28 semanas, equivalente a 140 días o 560 horas.
A modo de resumen:
\begin{multicols}{2}
\begin{itemize}[itemsep=0.25px]
    \item Semanas totales: 28.
    \item Días totales: 140.
    \item Horas totales: 560.
    \item Semana inicio: 2.
    \item Día inicio: 2024-01-08.
    \item Semana fin: 30.
    \item Duración fase 0 (semanas): 2.
    \item Duración fase 1 (semanas): 3.
    \item Duración fase 2 (semanas): 14.
    \item Duración fase 3 (semanas): 3.
    \item Duración fase 4 (semanas): 6.
\end{itemize}
\end{multicols}

\pagebreak
\section{Tareas}

\begin{center}
    \begin{table}[h]
        \centering
        \begin{tabular}{ l | l l l l }
            \hline
            ID & Tarea & Duración & Comienzo & Fin \\
            \hline
            001 & Planificación y estudio de viabilidad del TFG & 0.5 semanas & 2 & 3 \\
            002 & Búsqueda y estudio de bibliografía (JSP) & 0.5 semanas & 3 & 3 \\
            003 & Búsqueda y estudio de bibliografía (A*) & 0.5 semanas & 3 & 4 \\
            004 & Búsqueda y estudio de bibliografía (FPGA) & 0.5 semanas & 4 & 4 \\
            \hline
            101 & Diseño de programa orientado a objetos & 0.5 semanas & 4 & 5 \\
            102 & Diseño de juego de pruebas Python & 0.5 semanas & 5 & 5 \\
            103 & Codificación de versión Python & 1 semana & 5 & 6 \\
            104 & Corrección de defectos Python & 1 semana & 6 & 7 \\
            \hline
            201 & Diseño de juego de pruebas C++ & 0.5 semanas & 7 & 8 \\
            202 & Codificación de versión C++ (monohilo) & 4 semanas & 8 & 12 \\
            203 & Corrección de defectos C++ (monohilo) & 1 semana & 12 & 13 \\
            204 & Codificación de versión C++ (FCFS) & 0.5 semanas & 13 & 13 \\
            205 & Corrección de defectos C++ (FCFS) & 1 semana & 13 & 14 \\
            206 & Codificación de versión C++ (Batch) & 0.5 semanas & 14 & 15 \\
            207 & Corrección de defectos C++ (Batch) & 1 semana & 15 & 16 \\
            208 & Codificación de versión C++ (Recursive) & 0.5 semanas & 16 & 16 \\
            209 & Corrección de defectos C++ (Recursive) & 1 semana & 16 & 17 \\
            210 & Codificación de versión C++ (HDA*) & 0.5 semanas & 17 & 18 \\
            211 & Corrección de defectos C++ (HDA*) & 1 semana & 18 & 19 \\
            212 & Codificación de scripts & 0.5 semanas & 19 & 19 \\
            213 & Ejecución de scripts & 1 semana & 19 & 20 \\
            214 & Estudio de resultados & 1 semana & 20 & 21 \\
            \hline
            301 & Codificación y síntesis de prototipo en FPGA & 0.5 semanas & 21 & 22 \\
            302 & Codificación y síntesis de A* en FPGA & 1 semana & 22 & 23 \\
            303 & Ejecución de programa FPGA & 0.5 semanas & 23 & 23 \\
            304 & Estudio de resultados & 1 semana & 23 & 24 \\
            \hline
            401 & Redacción de capítulos teóricos de la memoria & 1 semana & 24 & 25 \\
            402 & Redacción de capítulos técnicos de la memoria & 1 semana & 25 & 26 \\
            403 & Redacción de capítulos con resultados de la memoria & 1 semana & 26 & 27 \\
            404 & Redacción de conclusiones de la memoria & 1 semana & 27 & 28 \\
            405 & Corrección y revisiones de la memoria con el tutor & 2 semanas & 28 & 30 \\
            \hline
        \end{tabular}
        \caption{Planificación del TFG.}
    \end{table}
\end{center}
