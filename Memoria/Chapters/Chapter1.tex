\Chapter{Problema a investigar}{Objetivos}

\section{Job Shop Scheduling Problem}

Se estudiará la implementación de una solución al problema
\italic{Job Shop Scheduling (JSP)}~\cite{Yan77}
\footnote{El problema también es conocido por otros nombres similares
como \italic{Job Shop Scheduling Problem (JSSP)} o Job Scheduling Problem (JSP).}.
Como su nombre indica, se trata de un problema en el que se debe
crear una planificación.
Desde el punto de vista de la ingeniería informática,
el JSP es un problema de optimización.

El problema JSP busca una planificación para una serie de
máquinas (o trabajadores) que deben realizar un número conocido
de trabajos.
Cada trabajo está formado por una serie de operaciones (o tareas),
con una duración conocida.
Las tareas de un mismo trabajo deben ser ejecutadas en un orden específico.

Existen numerosas variantes de este problema,
entre ellas existen variantes que permiten la ejecución
en paralelo de algunas tareas o requieren que alguna tarea
en específico sea ejecutada por un trabajador (o tipo de trabajador)
en particular.
Por ello, es clave denotar las normas que se aplicarán a la hora de resolver
el problema JSP\@:

\begin{enumerate}[itemsep=0.25px]
    \item Existen un número natural conocido de trabajos.
    \item Todos los trabajos tienen el mismo número natural conocido de tareas.
    \item A excepción de la primera tarea de cada trabajo,
    todas tienen una única tarea predecesora que debe ser completada
    antes de iniciar su ejecución.
    \item Cada tarea puede tener una duración distinta.
    \item La duración de cada tarea es un número natural conocido.
    \item Existe un número natural conocido de trabajadores.
    \item Todos los trabajadores son equivalentes,
    cualquiera puede realizar cualquier tarea.
    \item Una vez iniciada una tarea, no se puede interrumpir su ejecución.
    \item Un mismo trabajador puede intercalar la ejecución de tareas de diferentes trabajos.
    \item Un trabajador sólo puede realizar una tarea al mismo tiempo.
    \item Los tiempos de preparación de un trabajador antes de realizar una tarea son nulos.
    \item Los tiempos de espera entre la realización de una tarea y otra son nulos.
\end{enumerate}

\section{Método a resolver}

\subsection{Algoritmo}

Existen numerosos algoritmos capaces de resolver el problema del JSP\@.
Estrategias más simples como listas ordenadas en función de la duración
de los trabajos, algoritmos genéticos, técnicas gráficas,
algoritmos \italic{Branch and Bound} y heurísticos.
En este estudio se utilizará un algoritmo heurístico, A* (\italic{A star})~\cite{HNR68}
para resolver el problema JSP\@.

El algoritmo A* utiliza varios componentes para resolver problemas
de optimización. A continuación se describe cada uno de ellos.

\subsubsection{Componentes}

\paragraph{Estado}~

El estado es una estructura de datos que describe la situación
del problema en un punto determinado.
Estos estados deben ser comparables,
debe ser posible dados dos estados conocer si son iguales o distintos.
En el caso del JSP, el estado podría estar formado por lo siguiente:

\begin{itemize}[itemsep=0.25px]
    \item Instante de tiempo en el que comienza cada tarea planificada.
    \item Instante de tiempo futuro en el que cada trabajador estará libre.
\end{itemize}

El resultado final del JSP será un estado donde todas las tareas han sido planificadas.

\paragraph{Costes}~

El algoritmo A* utiliza 3 costes distintos para resolver el problema de optimización:

\subparagraph{Coste G}~

El coste G (de ahora en adelante $cost_g$) es el coste desde el estado inicial
hasta el estado actual.
Este coste es calculado buscando el mayor tiempo de fin de
las tareas ya planificadas.

\subparagraph{Coste H}~

El coste H (de ahora en adelante $cost_h$) es el coste estimado desde el estado actual
hasta el estado final.
Este coste es calculado utilizando una función heurística,
que estima el coste.
Esta función debe no debe ser optimista,
por lo que en este estudio se utilizará el sumatorio de duraciones
de las tareas restantes por planificar.

\subparagraph{Coste F}~

El coste F (de ahora en adelante $cost_f$) es el coste estimado desde el estado inicial
hasta el estado final pasando por el estado actual.

Por lo tanto, \[cost_f = cost_g + cost_h\]

\paragraph{Generación de sucesores}~

El algoritmo A* debe generar un número de estados sucesores dado un estado actual.
Por lo que será necesario una función que dado un estado retorne un listado de estados.

Dado un estado donde existen $N$ tareas por ejecutar ($T_0 \dots T_N$) y
un trabajador cualquiera sin tarea asignada tendrá $N$ estados sucesores.
En cada uno, el trabajador libre tendrá asignada cada una de las tareas,
desde $T_0$ hasta $T_N$.

\paragraph{Listas de prioridad}~

El algoritmo A* utiliza dos listas de estados: la lista abierta y la lista cerrada.
La lista cerrada contiene los estados que ya han sido estudiados mientras que
la lista abierta contiene los estados que aún están por estudiar.

Cada vez que se estudia un estado de la lista abierta,
se obtienen sus sucesores que son añadidos a la lista abierta
(siempre y cuando no estén en la lista cerrada)
mientras que el estado estudiado pasa a la lista cerrada.

Los estados de la lista abierta están ordenados en función de su $cost_f$,
de menor a mayor.
De esta forma, se tiene acceso inmediato al elemento con menor $cost_f$.

\subsubsection{Pseudocódigo}

\begin{lstlisting}[language=Python]
    lista_abierta = SortedList()
    lista_abierta.append(estado_inicial)

    g_costes = {}
    f_costes = {}

    g_costes[estado_inicial] = 0
    f_costes[estado_inicial] = calcular_h_coste(estado_inicial)

    while (not lista_abierta.empty()):
        estado_actual = lista_abierta.pop()

        if (estado_actual == estado_final):
            return estado_actual

        estados_sucesores = calcular_sucesores(estado_actual)

        for estado_sucesor in estados_sucesores:
            sucesor_g_coste = calcular_g_coste(estado_sucesor)
            if (sucesor_g_coste < g_costes[estado_sucesor]):
                g_costes[estado_sucesor] = sucesor_g_coste
                f_costes[estado_sucesor] = sucesor_g_coste + calcular_h_coste(estado_sucesor)
                if (estado_sucesor not in lista_abierta):
                    lista_abierta.append(estado_sucesor)
            
\end{lstlisting}

\subsection{Equipo de Estudio}

Este algoritmo es implementado y optimizado en diversas arquitecturas.
Posteriormente, se realizan comparaciones entre ellas.

\subsubsection{Arquitectura x86}

Inicialmente, se realiza una implementación del problema utilizando \Python.
Esta versión permite comprobar rápidamente el correcto funcionamiento del mismo
así como llevar a cabo pruebas rápidas sin necesidad de compilación y
estudiar los posibles cuellos de botella del algoritmo.

Posteriormente, se desarrolla una nueva versión del mismo algoritmo
utilizando C, un lenguaje compilado con mayor rendimiento
que facilita la paralelización gracias a librerías como 
\href{https://www.openmp.org/}{OpenMP}\@.

Una vez desarrolladas,
se realizan comparativas entre las distintas implementaciones
de esta aquitectura, en particular:

\begin{enumerate}[itemsep=0.25px]
    \item Comparativa entre el rendimiento monohilo de Python y C.
    \item Comparativa entre el rendimiento monohilo de C y multihilo de C.
\end{enumerate}

\subsection{FPGA}

Finalmente, se desarrolla una implementación del algoritmo
diseñado para ser ejecutado en una FPGA\@.
Esta aceleradora, se encuentra embebida en una placa SoC
Zybo Z7 10 acompañada de un procesador ARM\@.

Para realizar esta implementación,
se utiliza el software propio de Xilinx (AMD),
Vitis HDL\@.
Este programa ofrece entre muchas otras herramientas
un sintetizador capaz de transpilar código C a Verilog
que puede ser entonces compilado
para ejecutarse en la FPGA\@.

\section{Método de comparativas}

Las comparativas entre las diferentes implementaciones
del algoritmo se realizan en base a varias características.
Principalmente:

\begin{enumerate}[itemsep=0.25px]
    \item Tiempo de ejecución.
    \item Consumo de memoria volátil.
\end{enumerate}

Como es lógico, el algoritmo es ejecutado utilizando
varios datos de entrada distintos múltiples veces.
Es de crucial importancia tener en cuenta la
caché de los dispositivos,
esto implica que la primera ejecución de cada conjunto de datos
de entrada tiene un peor rendimiento que las siguientes.
Para aislar este hecho, las métricas de la
primera ejecución de cada conjunto de datos son ignoradas.
