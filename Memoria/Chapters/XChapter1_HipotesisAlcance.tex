\Chapter{Hipótesis de Partida y Alcance}{}
\index{Hipótesis de Partida y Alcance}

\section{Hipótesis de partida}
\index{Hipótesis de Partida y Alcance!Hipótesis de partida}

El presente proyecto tiene como objetivo principal descubrir los
beneficios del paralelismo aplicado al algoritmo A*.
Para estudiar el rendimiento de las implementaciones desarrolladas
se utilizará el Job Shop Scheduling Problem.
Adicionalmente, se observará el rendimiento de una implementación
de la misma solución en una FPGA.

La motivación principal para la realización de este proyecto
emana de un interés personal por el diseño, implementación y optimización
de algoritmos aplicables a problemas reales.
Adicionalmente, analizar la utilidad y límites de un dispositivo como
las FPGAs, frecuentemente utilizado
en entornos industriales donde los problemas
de optimización son comunes pero suelen ser resueltos utilizando
CPUs tradicionales en servidores.

El Job Shop Scheduling Problem es un problema de optimización
sobre la planificación de horarios.
Este es un problema np-hard mundialmente conocido
por la comunidad científica,
ha sido resuelto utilizando un gran abanico de algoritmo diferentes
y está profundamente estudiado.

La resolución del Job Shop Scheduling Problem requiere el diseño e implementación de
un algoritmo capaz de recibir como entrada las descripciones de una plantilla
de trabajadores, un listado de trabajos y tareas a realizar. Puede ser
aplicable en ámbitos industriales donde la automatización de la creación
de planificaciones sea de interés.

\pagebreak

\section{Alcance}
\index{Hipótesis de Partida y Alcance!Alcance}

El presente documento describe el problema a resolver en detalle,
los diferentes algoritmos implementados,
observaciones sobre los mismos,
casos de prueba utilizados para obtener métricas de rendimiento
y observaciones sobre las métricas obtenidas.

Entre las observaciones tanto de los algoritmos como de las métricas
obtenidas, se encontrarán razonamientos sobre los resultados así
como explicaciones de las razones por las cuales un algoritmo
presenta un rendimiento distinto a otro.
