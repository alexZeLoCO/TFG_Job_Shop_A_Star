\Chapter{Hipótesis de Partida y Alcance}{}
\index{Hipótesis de Partida y Alcance}

\section{Hipótesis de partida}
\index{Hipótesis de Partida y Alcance!Hipótesis de partida}

Existen una infinidad de algoritmos con distintos diseños, propósitos y utilidades.
Como es de esperar, algunos requieren más tiempo para ser ejecutados que otros
pero a diferencia de lo que muchos se podrían esperar,
obtener un algoritmo que no se pueda resolver en un tiempo razonable
es más simple de lo que parece.

La complejidad de un algoritmo se refiere a la diferencia
en su tiempo de ejecución en función del tamaño de su entrada.
Entonces, la complejidad es la primera derivada del tiempo
de ejecución.

Los algoritmos se pueden clasificar en dos categorias, P y NP.
Aquellos pertenecientes a la primera categoria se pueden resolver
en un tiempo polinomial,
mientras que para los problemas NP no se conoce una solución
con complejidad polinomial.

Frecuentemente estos problemas son resueltos utilizando
tamaños de entrada muy pequeños o algoritmos
que den un resultado aproximado que es posteriormente
comparado con el tiempo necesario para obtenerlo.
Adicionalmente, estos algoritmos se pueden aprovechar del
paralelismo utilizando varios núcleos de una CPU
o de hardware específico como GPGPUs.

Resolver este tipo de problemas es de interés en ámbitos
industriales donde las soluciones podrían ser utilizadas
para optimizar sus procesos o la creación de nuevos productos.
En estos entornos industriales es común el uso de
hardware específico como las FPGA u otros ASIC.

El presente proyecto tiene como objetivo principal 
explorar los límites del A*, uno de los algoritmos más
conocidos con aplicaciones en estos ámbitos industriales
y descubrir los beneficios del paralelismo aplicables
a este algoritmo.
Adicionalmente, se observará el rendimiento de una implementación
de la misma solución en una FPGA,
prestando particular atención a los límites
de este hardware en términos de qué tamaños de problema
se pueden resolver en un tiempo razonable.

\pagebreak

\section{Alcance}
\index{Hipótesis de Partida y Alcance!Alcance}

El presente documento describe el problema a resolver en detalle,
los diferentes algoritmos implementados,
observaciones sobre los mismos,
casos de prueba utilizados para obtener métricas de rendimiento
y observaciones sobre las mismas.

Entre las observaciones tanto de los algoritmos como de las métricas
obtenidas, se encontrarán razonamientos sobre los resultados así
como explicaciones de las razones por las cuales un algoritmo
presenta un rendimiento distinto a otro.

El Job Shop Scheduling Problem es un problema de optimización
sobre la planificación de horarios.
Este es un problema np-hard mundialmente conocido
por la comunidad científica,
ha sido resuelto utilizando un gran abanico de algoritmo diferentes
y está profundamente estudiado.

La resolución del Job Shop Scheduling Problem requiere el diseño e implementación de
un algoritmo capaz de recibir como entrada las descripciones de una plantilla
de trabajadores, un listado de trabajos y tareas a realizar. Puede ser
aplicable en ámbitos industriales donde la automatización de la creación
de planificaciones sea de interés.
