\Chapter{Metodología de trabajo}{Método de resolución y comparativas}
\index{Metodología de trabajo}

\section{Método de resolución}
\index{Metodología de trabajo!Método de resolución}

\subsection{El Algoritmo A*}
\index{Algoritmo A*}
\label{ssec:AlgoritmoA*}

Existen numerosos algoritmos capaces de resolver el problema del JSP\@.
Estrategias más simples como listas ordenadas en función de la duración
de los trabajos, algoritmos genéticos, técnicas gráficas,
algoritmos \italic{Branch and Bound} y heurísticos.
En este estudio se utilizará un algoritmo heurístico, A* (\italic{A star})~\cite{HNR68}
para resolver el problema JSP\@.

El A* es una evolución del algoritmo de Dijkstra.
Su principal diferencia es la implementación de una función heurística
que se utiliza para decidir el siguiente nodo a expandir.
De esta forma, se podría decir que el algoritmo A* va `guiado'
hacia la solución, mientras que el algoritmo de Dijkstra
sigue los caminos con menor coste (Véase figura \ref{fig:DijkstraA*Comparison}).

\begin{figure}[h]
\begin{subfigure}{.5\textwidth}
\begin{center}
\begin{tikzpicture}[node distance=1.5cm]
    \node (00) [dot, fill=green!30] {00};
    \node (01) [dot, right of=00, fill=blue!30] {01};
    \node (02) [dot, right of=01] {02};
    \node (03) [dot, right of=02, fill=blue!30] {03};

    \node (10) [dot, below of=00] {10};
    \node (11) [dot, right of=10, fill=blue!30] {11};
    \node (12) [dot, right of=11, fill=blue!30] {12};
    \node (13) [dot, right of=12, fill=blue!30] {13};

    \node (20) [dot, below of=10] {20};
    \node (21) [dot, right of=20] {21};
    \node (22) [dot, right of=21, fill=blue!30] {22};
    \node (23) [dot, right of=22] {23};

    \node (30) [dot, below of=20] {30};
    \node (31) [dot, right of=30, fill=blue!30] {31};
    \node (32) [dot, right of=31, fill=blue!30] {32};
    \node (33) [dot, right of=32, fill=yellow!30] {33};

    \draw [arrow] (00) -- (11);
    \draw [arrow] (00) -- (01);

    \draw [arrow] (11) -- (22);
    \draw [arrow] (11) -- (12);
    \draw [arrow] (12) -- (13);
    \draw [arrow] (13) -- (03);

    \draw [arrow] (22) -- (33);
    \draw [arrow] (22) -- (32);

    \draw [arrow] (32) -- (31);
\end{tikzpicture}
\subcaption{Algoritmo Dijkstra}
\end{center}
\end{subfigure}
\begin{subfigure}{.5\textwidth}
\begin{center}
\begin{tikzpicture}[node distance=1.5cm]
    \node (00) [dot, fill=green!30] {00};
    \node (01) [dot, right of=00] {01};
    \node (02) [dot, right of=01] {02};
    \node (03) [dot, right of=02] {03};

    \node (10) [dot, below of=00] {10};
    \node (11) [dot, right of=10, fill=blue!30] {11};
    \node (12) [dot, right of=11] {12};
    \node (13) [dot, right of=12] {13};

    \node (20) [dot, below of=10] {20};
    \node (21) [dot, right of=20] {21};
    \node (22) [dot, right of=21, fill=blue!30] {22};
    \node (23) [dot, right of=22] {23};

    \node (30) [dot, below of=20] {30};
    \node (31) [dot, right of=30] {31};
    \node (32) [dot, right of=31] {32};
    \node (33) [dot, right of=32, fill=yellow!30] {33};

    \draw [arrow] (00) -- (11);
    \draw [arrow] (11) -- (22);
    \draw [arrow] (22) -- (33);
\end{tikzpicture}
\subcaption{Algoritmo A*}
\end{center}
\end{subfigure}
\caption{Comparativa entre algoritmos Dijkstra y A*}
\label{fig:DijkstraA*Comparison}
\end{figure}

El algoritmo A* utiliza varios componentes para resolver problemas
de optimización. A continuación se describe cada uno de ellos.

\pagebreak

\subsubsection{Pseudocódigo}
\index{Algoritmo A*!Pseudocódigo}

\begin{lstlisting}[language=Python, caption=Pseudocódigo del algoritmo A*, label=lst:PseudocodigoA*]
    lista_abierta = SortedList()
    lista_abierta.append(estado_inicial)

    g_costes = {estado_inicial: 0}
    f_costes = {estado_inicial: calcular_h_coste(estado_inicial)}

    while (not lista_abierta.empty()):
        estado_actual = lista_abierta.pop()

        if (estado_actual == estado_final):
            return estado_actual

        estados_sucesores = calcular_sucesores(estado_actual)

        for estado_sucesor in estados_sucesores:
            sucesor_g_coste = calcular_g_coste(estado_sucesor)
            if (sucesor_g_coste < g_costes[estado_sucesor]):
                g_costes[estado_sucesor] = sucesor_g_coste
                f_costes[estado_sucesor] = sucesor_g_coste + calcular_h_coste(estado_sucesor)
                if (estado_sucesor not in lista_abierta):
                    lista_abierta.append(estado_sucesor)
\end{lstlisting}

El algoritmo (véase pseudocódigo \ref{lst:PseudocodigoA*}) utiliza una lista de estados a explorar
inicialmente compuesta por el estado inicial (ln1-2).
Además, se crean estructuras de datos diseñadas para almacenar
el mejor coste conocido para llegar a un estado cualquiera (ln4-5).
De esta forma, \lstinline{g_costes[estado_a]} es el mejor coste G
conocido hasta el momento para el \lstinline{estado_a}.

El resto del algoritmo se encarga de explorar los nodos
de la lista hasta que esté vacía o el estado actual sea el objetivo.
Si el estado actual es el objetivo, el algoritmo retorna (ln13).
En otro caso, calcula los sucesores del estado actual
y los procesa individualmente (ln16).
Si el coste G del sucesor es mejor que el mejor coste G
conocido para llegar al estado sucesor
se graban sus costes y se añade a la lista
si no estaba ya en ella (ln19-24).

\pagebreak

\subsubsection{Componentes A*}

\paragraph{Estado}~
\index{Algoritmo A*!Estado}

El estado es una estructura de datos que describe la situación
del problema en un punto determinado.
Estos estados deben ser comparables,
debe ser posible dados dos estados conocer si son iguales o distintos.
En el caso del JSP, el estado representa una planificación parcial o completa
en la cual una serie de tareas han sido planificadas:

\begin{itemize}[itemsep=0.25px]
    \item Instante de tiempo en el que comienza cada tarea planificada.
    \item Instante de tiempo futuro en el que cada trabajador estará libre
    (i.e. finaliza la tarea que estaba realizando).
\end{itemize}

El resultado final del JSP será un estado donde todas las tareas han sido planificadas.

\paragraph{Costes}~
\index{Algoritmo A*!Costes}

El algoritmo A* utiliza 3 costes distintos para resolver el problema de optimización:

\subparagraph{Coste G}~
\index{Algoritmo A*!Costes!Coste G}

El coste G (de ahora en adelante $cost_g$) es el coste desde el estado inicial
hasta el estado actual.
Este coste es calculado buscando el mayor tiempo de fin de
las tareas ya planificadas.

\subparagraph{Coste H}~
\index{Algoritmo A*!Costes!Coste H}

El coste H (de ahora en adelante $cost_h$) es el coste estimado desde el estado actual
hasta el estado final.
Este coste es calculado utilizando una función heurística,
que estima el coste.

Idealmente esta función heurística será una cota inferior cuando el problema
sea de minimización y una cota superior cuando sea de maximización.
El heurístico deberá ser entonces diseñado atendiendo al tipo de problema.

\subparagraph{Coste F}~
\index{Algoritmo A*!Costes!Coste F}

El coste F (de ahora en adelante $cost_f$) es el coste estimado desde el estado inicial
hasta el estado final pasando por el estado actual.

Por lo tanto, \[cost_f = cost_g + cost_h\]

\pagebreak

\paragraph{Generación de sucesores}~
\index{Algoritmo A*!Generación de sucesores}

El algoritmo A* debe generar un número de estados sucesores dado un estado cualquiera.
Por lo que será necesario una función que dado un estado retorne este listado de estados.

Un estado con $N$ trabajos por completar ($J_0 \dots J_N$)
tiene $N$ tareas ($T_0 \dots T_N$) que están listas para ser ejecutadas
(sus predecesoras han sido ya completadas).
Este estado generará entonces $N$ estados sucesores
donde en cada uno se planificará una de las tareas
para el primer instante de tiempo posible.
Este primer instante se calcula utilizando el
máximo entre el instante de fin de la tarea predecesora y
el instante en el que el trabajador asignado
para la tarea a planificar $T_x$ queda libre.

\begin{examplebox}
    En el ejemplo anterior (\ref{ssec:A*Example}) con trabajos:
    \begin{enumerate}[start=0, itemsep=0.25px]
        \item (2, 0), (5, 1), (1, 2)
        \item (3, 1), (3, 2), (3, 0)
    \end{enumerate}
    El estado:
    \begin{itemize}[itemsep=0.25px]
        \item Planificación:
        \begin{enumerate}[start=0, itemsep=0.25px]
            \item (0, 3, -1)
            \item (0, -1, -1)
        \end{enumerate}
        \item Estado trabajadores: (2, 8, 0)
    \end{itemize}
    Tiene dos trabajos sin completar, por lo que generará dos sucesores.
    Uno en el que se planifica la primera tarea sin planificar del trabajo 0.
    \begin{itemize}[itemsep=0.25px]
        \item Instante en el que se puede iniciar la tarea: $3 + 5 = 8$
        \item Instante en el que el trabajador está libre: $0$
    \end{itemize}
    Por lo tanto, la tarea comenzará en $max(0, 8) = 8$.
    Y el trabajador estará libre en $8 + 1 = 9$.
    \begin{itemize}[itemsep=0.25px]
        \item Planificación:
        \begin{enumerate}[start=0, itemsep=0.25px]
            \item (0, 3, 8)
            \item (0, -1, -1)
        \end{enumerate}
        \item Estado trabajadores: (2, 8, 9)
    \end{itemize}

    Y otro en el que se planifica la primera tarea sin planificar del trabajo 1.
    \begin{itemize}[itemsep=0.25px]
        \item Instante en el que se puede iniciar la tarea: $0 + 3 = 3$
        \item Instante en el que el trabajador está libre: $0$
    \end{itemize}
    Por lo tanto, la tarea comenzará en $max(0, 3) = 3$.
    Y el trabajador estará libre en $3 + 3 = 6$.
    \begin{itemize}[itemsep=0.25px]
        \item Planificación:
        \begin{enumerate}[start=0, itemsep=0.25px]
            \item (0, 3, -1)
            \item (0, 3, -1)
        \end{enumerate}
        \item Estado trabajadores: (2, 8, 6)
    \end{itemize}
 
\end{examplebox}

\paragraph{Listas de prioridad}~
\index{Algoritmo A*!Listas de prioridad}

El algoritmo A* utiliza dos listas de estados: la lista abierta y la lista cerrada.
La lista cerrada contiene los estados que ya han sido estudiados mientras que
la lista abierta contiene los estados que aún están por estudiar.

Cada vez que se estudia un estado de la lista abierta,
se obtienen sus sucesores que son añadidos a la lista abierta
(siempre y cuando no estén en la lista cerrada)
mientras que el estado estudiado pasa a la lista cerrada.

Los estados de la lista abierta están ordenados en función de su $cost_f$,
de menor a mayor.
De esta forma, se tiene acceso inmediato al elemento con menor $cost_f$.

\pagebreak

\subsection{Equipo de Estudio}
\index{Equipo de Estudio}

Este algoritmo es implementado y optimizado en diversas arquitecturas.
Posteriormente, se realizan comparaciones entre ellas.

\subsubsection{Arquitectura x86}
\index{Equipo de Estudio!Arquitectura x86}

Inicialmente, se realiza una implementación del algoritmo utilizando \Python.
Esta versión permite comprobar rápidamente el correcto funcionamiento del mismo
así como llevar a cabo pruebas rápidas sin necesidad de compilación y
estudiar los posibles cuellos de botella del algoritmo.

Posteriormente, se desarrolla una nueva versión del mismo algoritmo
utilizando C++, un lenguaje compilado, imperativo y orientado a objetos
que facilita la paralelización gracias a librerías como 
\href{https://www.openmp.org/}{OpenMP}\@.

Una vez desarrolladas ambas versiones monohilo,
se comienza la implementación de versiones multihilo
que serán posteriormente comparadas.

\subsubsection{FPGA}
\index{Equipo de Estudio!FPGA}

Finalmente, se desarrolla una implementación del algoritmo
diseñado para ser ejecutado en una FPGA\@.
Esta aceleradora, se encuentra embebida en una placa SoC
Zybo Z7 10 acompañada de un procesador ARM\@.

Para realizar esta implementación,
se utiliza el software propio de Xilinx (AMD),
Vitis HLS, Vitis (Vitis Unified) y Vivado\@.
Este programa ofrece entre muchas otras herramientas
un sintetizador capaz de transpilar código C++ a VHDL o Verilog
que puede ser entonces compilado
para ejecutarse en la FPGA\@.

\pagebreak
\section{Método de comparativas}

Las comparativas entre las diferentes implementaciones
del algoritmo se realizan en base a varias características.
Principalmente:

\begin{enumerate}[start=0, itemsep=0.25px]
    \item Tiempo de ejecución.
    \item Calidad de la solución.
\end{enumerate}

Como es lógico, el algoritmo es ejecutado utilizando
distintos datos de entrada múltiples veces.

\begin{notebox}
    La segunda ejecución de cualquier algoritmo suele tender a
    requerir menos tiempo debido al funcionamiento de la caché.
    
    Para evitar este fenómeno, el algoritmo se ejecuta
    siempre $N$ veces ignorando las métricas de la primera ejecución.
\end{notebox}

